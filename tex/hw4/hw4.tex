%%
% homework template
%


\documentclass{article}

%\usepackage{enumitem}
\usepackage[utf8]{inputenc}
\usepackage{graphicx}
\usepackage{amssymb}
\usepackage{amsmath}
\usepackage{epstopdf}
\usepackage{enumerate}
\usepackage{booktabs}
\usepackage{multirow}
\usepackage[usenames,dvipsnames,svgnames,table]{xcolor}
\usepackage{url}
\usepackage{mathtools}

\usepackage{dsfont}
\usepackage{tikz}
\usetikzlibrary{trees}
\usetikzlibrary{arrows,positioning,shapes.multipart}
\usepackage{makecell}
\usepackage{hyperref}
\usepackage{cleveref}
\usepackage{listings}
\usepackage{bbm}
\usepackage{bm}
\usepackage{tikz}
\usetikzlibrary{trees}
\usepackage{makecell}
\usepackage{listings}
\usepackage{bbm}
\usepackage{bm}
\usepackage{xspace}

\usepackage{tcolorbox}
\tcbuselibrary{breakable} % to allow boxes to break across pages

% Define a new answer box environment with an optional height parameter.
\newtcolorbox{answerbox}[1][]{%
  breakable,
  colback=white,       % Background color
  colframe=black,      % Border color
  boxrule=0.5pt,       % Border thickness
  sharp corners,       % Box style
  before skip=10pt,    % Space before the box
  after skip=10pt,     % Space after the box
  % Allow adjustable height via an optional parameter (default: 3cm)
  height=\dimexpr\ifx#1\empty 3cm\else #1\fi\relax,
}

\usetikzlibrary{arrows,positioning,shapes.multipart}

\graphicspath{{./hw4/}}


\newcommand{\class}{ CS58700-DPL Spring 2025 }
\newcommand{\website}{{\tiny\url{https://www.cs.purdue.edu/homes/ribeirob/courses/Spring2025/index.html}}}
\newcommand{\homeworknumber}{4}
\newcommand{\duedate}{ {\bf 6:00pm}, April 20th (open until 6:00am April 21st)}
\newcommand{\code}{CODECODECODE}

%\pagestyle{headings}
\setlength{\parskip}{1pc}
\setlength{\parindent}{0pt}
\setlength{\topmargin}{-1pc}
\setlength{\textheight}{8.5in}
\setlength{\oddsidemargin}{0pc}
\setlength{\evensidemargin}{0pc}
\setlength{\textwidth}{6.5in}

\newcommand{\var}{\mbox{var}}
\newcommand{\cov}{\mbox{cov}}

\newcommand{\mA}{\bm{A}}
\newcommand{\mH}{\bm{H}}
\newcommand{\mW}{\bm{W}}
\newcommand{\mX}{\bm{X}}
\newcommand{\mY}{\bm{Y}}

\newcommand{\vb}{\bm{b}}
\newcommand{\vc}{\bm{c}}
\newcommand{\vh}{\bm{h}}
\newcommand{\vm}{\bm{m}}
\newcommand{\vx}{\bm{x}}
\newcommand{\vz}{\bm{z}}

\newcommand{\train}{\text{(tr)}}
\newcommand{\test}{\text{(te)}}
\newcommand{\source}{\text{(source)}}
\newcommand{\target}{\text{(target)}}


\newcommand*\dbar[1]{\overline{\overline{\lower0.2ex\hbox{$#1$}}}}
\newcommand{\harrow}[1]{\mathstrut\mkern2.5mu#1\mkern-11mu\raise1.6ex\hbox{$\scriptscriptstyle\rightharpoonup$}}

\newcommand{\newpart}{
\stepcounter{partno}
\noindent
{\bf (\alph{partno})}
}

\newcommand{\header}{
\newpage
\noindent
\framebox{ \vbox{\class Homework \hfill --- Homework \homeworknumber --- \hfill  Last update: \today
  \\ \website \hfill {\color{red} Due: \duedate} }}
\bigskip
\newline
{\bf Instructions and Policy:} Each student should write up their own solutions independently, no copying of any form is allowed.
You MUST indicate the names of the people you discussed a problem with; ideally you should
discuss with no more than two other people.\\
{\color{red} YOU MUST INCLUDE YOUR NAME IN THE HOMEWORK}\\
You need to submit your answer in PDF. {\LaTeX }  is typesetting is encouraged but not required.  Please write clearly and concisely - clarity and
brevity will be rewarded. Refer to known facts as necessary. 
\newline


}

\newcounter{questionno}
\setcounter{questionno}{-1}
\newcounter{partno}

\newcommand{\question}[1]{
\noindent
\newline
\stepcounter{questionno}
\setcounter{partno}{0}
{\bf Q\arabic{questionno} (#1 pts): }
}


%% -=-=-=-=-=-=-=-=-=-=-=-=-=-=-=-=-=-=-=-=-=-=-=-=-=-=- %%


\begin{document}

% Common homework head context.
%
\header
\question{%
    {\color{red}0pts correct answer, -1,000pts incorrect answer: (0,-1,000)}%
}
%
A correct answer to the following questions is worth 0pts.
%
An incorrect answer is worth -1,000pts, which carries over to other homeworks
and exams, and can result in an F grade in the course.
%
\begin{enumerate}[(1)]
\item Student interaction with other students / individuals:
\begin{enumerate}[(a)]
\item I have copied part of my homework from another student or another person (plagiarism).
\item \textbf{Yes, I discussed the homework with another person but came up with my own answers. Their name(s) is (are): Jason Fotso}
\item No, I did not discuss the homework with anyone
\end{enumerate}
\item On using online resources:
\begin{enumerate}[(a)]
\item I have copied one of my answers directly from a website (plagiarism).
\item \textbf{I have used online resources to help me answer this question, but I came up with my own answers (you are allowed to use online resources as long as the answer is your own). Here is a list of the websites I have used in this homework:\\ \url{https://pytorch.org/docs/main}}
\item I have not used any online resources except the ones provided in the course website.
\end{enumerate}

\end{enumerate}

\newpage

\section*{Homework \homeworknumber }

\subsection*{Q1: Conceptual Questions (4.5 pts)}

\begin{enumerate}



\item ({\bf 1.0 pts}) Detail how we can replace the equivariant representation of the self-attention part of the Transformer architecture with an {\em equivariant representation} using a MLP architecture with input and output dimensions $n\times d$, where $n$ is the number of elements in the sequence, and $d$ is the hidden dimension. Since the MLP can only take a fixed size input, for simplicity, we consider every input sequence to have exactly $n$ tokens.
Specifically, replace the representation of the $i$-th word of the $m$-th self-attention head,  $z^{(m)}_i$, of our Transformer lecture with a corresponding MLP output.\\
{\bf Hint:}  Make sure the representation coming out of the MLP, $h_1,\ldots,h_n$ for a $n$-th word sentence, is equivariant. The key is to use those set representation techniques we learned in lecture. In particular, your MLP architecture is allowed to shared weights for tokens at different locations achieve equivariance.
% Students can write or type their answer here.

\begin{answerbox}[3.5in]
To construct an MLP-based architecture that replaces the self-attention
representation $z^{(m)}_i$ while maintaining permutation equivariance, we
proceed as follows.

Let the input be a sequence of $n$ tokens, $x_1, \ldots, x_n \in \mathbb{R}^d$,
and let $X \in \mathbb{R}^{n \times d}$ denote the full input.

Apply a shared MLP $\psi: \mathbb{R}^{2d} \to \mathbb{R}^d$ to each token. To incorporate context, we concatenate each token $x_i$ with a global summary vector $g \in \mathbb{R}^d$:
\[
g = \sum_{j=1}^n x_j
\quad \text{and} \quad
h_i = \psi([x_i; g]) \quad \text{for } i = 1, \ldots, n
\]
where $[x_i; g]$ denotes the concatenation of the two vectors.

The same function $\psi$ is applied at every position $i$, ensuring that the
output $h_1, \ldots, h_n$ is equivariant to permutations of the input sequence.

Equivariance is achieved by weight sharing and permutation-invariant aggregation
(via summation). Concatenating the global summary vector with each token allows
the MLP to approximate interaction between tokens (since there is no
finer-grained pair-wise attention mechanism).
\end{answerbox}

\item ({\bf 1.0 pt}) Consider a decoder-only transformer model designed for a text generation task. Suppose we want to replace the traditional Rotary Position Embeddings (RoPE) with a novel variant that incorporates an additional learnable parameter, which scales the rotational frequency.

Specifically, let's denote the embedding dimension as $d$, the sequence length as $n$, and the position in the sequence as $pos$. The standard RoPE can be represented as:
$$
\begin{aligned}
\text{ROPE}_{2i}(pos) &= \sin(\frac{pos}{10000^{2i/d}}) \\
\text{ROPE}_{2i+1}(pos) &= \cos(\frac{pos}{10000^{2i/d}})
\end{aligned}
$$

In our proposed variant, we introduce learnable parameters $\alpha_{i}$ that scales the rotational frequency as:
$$
\begin{aligned}
\text{ROPE}^\alpha_{2i}(pos) &= \sin(\alpha_{i} \cdot \text{pos}) \\
\text{ROPE}^\alpha_{2i+1}(pos) &= \cos(\alpha_{i} \cdot \text{pos})
\end{aligned}
$$

{\bf Assuming the learnable parameters $\alpha_{i}$ are optimized during training, answer the following question:} Provide a mathematical formulation of how the proposed RoPE variant affects the relative positions of tokens in the self-attention mechanism. Specifically, for which values of $(\alpha_0,\ldots)$ would the model almost lose the ability to encode some relative positional embeddings and what would be these positions?

\textbf{Hint:} In what ways might this modification negatively impact the model's performance on tasks that require accurately distinguishing between long-range and short-range dependencies? 

{\bf Support your arguments with the appropriate mathematical equations.}
~
\begin{answerbox}[3in]
There are two regimes in which the proposed RoPE variant may fail to encode relative position information:

\begin{itemize}
    \item \textbf{Small $\alpha_i$ (underfitting):} When $\alpha_i \ll \frac{1}{n}$, the rotation angle $\alpha_i \cdot (pos_1 - pos_2) \approx 0$ for all $pos_1, pos_2$, making the sinusoidal outputs nearly constant across all positions. This effectively collapses positional variation and prevents the model from distinguishing between any positions.

    \item \textbf{Large $\alpha_i$ (aliasing):} When $\alpha_i \cdot (pos_1 - pos_2) = 2\pi k$ for some integer $k$, distant positions become indistinguishable because the sine and cosine functions are periodic. This aliasing effect leads to incorrect modeling of long-range dependencies, as different positions are rotated to the same angle.

\end{itemize}

In both cases, the model loses the ability to encode accurate relative position information in the attention mechanism, degrading performance on tasks involving structured or long-range dependencies.
\end{answerbox}
~
\newpage

\item ({\bf 2.5 pt}) We want use to Low-Rank Adaptation (LoRA) fine-tune a large decoder-only transformer (Causal masking) to answer questions about a long document that is given at the beginning of the input sequence. Note that the input sequence consists of a long document followed by a prompt, requiring the model to attend back to the document when generating answers. The fine tuning can be described as operating on the training data $D = \{(x_i,y_i)\}_i$, where the $i$-th fine-tuning example, $x_i$ is the document + prompt sequence and $y_i$ is the answer sequence.

Each self-attention module contains original weight matrices $W_q$, $W_k$, and $W_v \in \mathbb{R}^{d_{model} \times d_{model}}$. LoRA decomposes each matrix into a fixed pre-trained component $W_0$ plus a low-rank update $W_{\text{LoRA}} = BA$, where $B \in \mathbb{R}^{d_{model} \times r}$ and $A \in \mathbb{R}^{r \times d_{model}}$ with $r \ll d_{model}$.

\subsection*{Questions}

\begin{enumerate}
    \item (1.0 pt) \textbf{Attention Pattern Analysis}:  
    Consider applying LoRA only to the last transformer block on either the query matrix ($W_q$) or key matrix ($W_k$). Explain which option would require significantly less computation to fine-tune. Support your reasoning with theoretical arguments about how query-key impacts the prediction of $y_i[0]$, the first token of $y_i$.\\
    {\bf Hint 1:} Consider what you can cache of the computation and perform only once.\\
    {\bf Hint 2:} The prediction of $y_i[0]$ depends only on the embedding of the last token of $x_i$.
~
\begin{answerbox}[2.5in]
    
Applying LoRA to the \textbf{query matrix} $W_q$ requires significantly less
computation than applying it to the key matrix $W_k$.

To predict the first output token $y_i[0]$, the model uses: a single query
vector, derived from the final token of $x_i$, but \textbf{all key vectors},
derived from the entire input $x_i$ (document + prompt), which can be long

If LoRA is applied to $W_q$, only one low-rank update needs to be computed per
example, since there is only one query. However, if LoRA is applied to $W_k$,
all key vectors (one per input token) must be recomputed with the new LoRA
parameters.

Moreover, keys from the input can be cached across examples if $W_k$ is fixed.
But when LoRA is applied to $W_k$, this caching is no longer valid, increasing
compute further.

Thus, applying LoRA to $W_q$ is computationally cheaper and more efficient
during fine-tuning under causal masking.
\end{answerbox}

~

\item (0.5 pts) \textbf{Value Matrix Adaptation}:  

Discuss now applying LoRA solely to the value matrix ($W_v$) of the last
transformer block. Highlight qualitative difference between fine tuning $W_v$
from fine tuning $W_q$. Propose an empirical experiment that could validate your
claims: define a task that would help you decide which procedure is best to find
information inside a long document.\\ {\bf Hint:} Assume that the model's
challenge lies in locating information to answer the query.

\begin{answerbox}[3in]
LoRA applied to $W_q$ improves how the model formulates queries, i.e., how it searches for relevant content in the input sequence. In contrast, LoRA applied to $W_v$ modifies what content is extracted once relevant positions have been identified via attention.

Thus, fine-tuning $W_q$ primarily improves the model's ability to find information, while fine-tuning $W_v$ enhances how that information is represented or retrieved after it is found.

To empirically compare them, we propose a long-document question answering task, where the model is given a long input article followed by a factual question. The correct answer appears only once in the article. We fine-tune separate models with LoRA applied to $W_q$ and $W_v$ respectively, and evaluate answer accuracy. This setup reveals whether improving the model’s ability to locate (via $W_q$) or extract (via $W_v$) information better supports the task.
\end{answerbox}

\item (1.0 pt) \textbf{LoRA Exclusively on the Last MLP}: Compare the effectiveness of applying LoRA exclusively to the last output Multi-Layer Perceptron (MLP) with the best approach from the previous analyses, where LoRA was applied to specific transformer weight matrices ($W_q$, $W_k$, or $W_v$) at the last layer. Discuss one advantage and one notable disadvantage of fine-tuning only the last MLP using LoRA, without adapting the transformer blocks. Specifically, elaborate on why restricting LoRA to the last MLP could potentially limit the model's capacity to fit the data $D$, considering factors such as: (a) the expressiveness of the output layer in capturing task-specific patterns, (b) the role of the transformer blocks in learning contextualized representations, and (c) the interplay between the adapted output layer and the frozen transformer weights in determining the model's overall performance on the downstream task. Support your reasoning with theoretical arguments about how the last output MLP impacts the prediction of $y_i[0]$, the first token of $y_i$.\\
{\bf Support your arguments with the appropriate mathematical equations.}

\begin{answerbox}[5in]
    Using LoRA exclusively on the last MLP has both advantages and disadvantages compared to applying it to the last transformer block's weight matrices.
\begin{itemize}
    \item \textbf{Advantage:} Fine-tuning only the last MLP is computationally efficient. The MLP has fewer parameters than the entire transformer block, so LoRA can be applied with less overhead. This allows for faster training and lower memory usage, making it suitable for resource-constrained environments.

    \item \textbf{Disadvantage:} Restricting LoRA to the last MLP limits the model's ability to learn task-specific patterns. The transformer blocks are crucial for learning contextualized representations of the input data. By not adapting them, we miss out on the rich interactions and dependencies captured in earlier layers. The last MLP can only transform the final representation, which may not be sufficient for complex tasks requiring deeper contextual understanding.

    The prediction of $y_i[0]$ depends on the final representation $h_n$ from the last transformer block:
    \[
    y_i[0] = \text{MLP}(h_n)
    \]
    If $h_n$ is not well-adapted to the task due to frozen transformer weights, the MLP may struggle to produce accurate predictions, as it lacks the necessary context and information from earlier layers.
    The expressiveness of the output layer is limited by the frozen transformer blocks, which may not capture the intricate relationships in the data. This can lead to suboptimal performance on downstream tasks, especially when the model needs to learn complex mappings from input to output.
    In summary, while applying LoRA to the last MLP is computationally efficient, it may hinder the model's ability to learn task-specific patterns and capture contextualized representations, ultimately affecting performance on downstream tasks.
\end{itemize}
% Students can write or type their answer here.
\end{answerbox}
~
\end{enumerate}



\end{enumerate}



\newpage

\subsection*{Q3: Transformer and Language Modeling (5.5 pts)}


In this homework, you are tasked with implementing a Transformer model for the causal language modeling task.

\hfill


\textbf{A GPU is essential for some of the tasks, thus make sure you follow the
instruction to set up the GPU environment on scholar.rcac.purdue.edu.}
%

% 
You can reuse your \texttt{CS587} conda environment created for previous homework. 
%
For this homework, you will need to install a new package, \texttt{tiktoken}:

{\color{red}
\begin{verbatim}
pip install tiktoken
\end{verbatim}
}

If you need to recreate the \texttt{CS587} environment, here are the commands:

\begin{verbatim}
module load anaconda/2024.02-py311
conda create -n CS587 python=3.11 ipython ipykernel -y
conda activate CS587
pip install torch==2.5.1 torchvision==0.20.1 --index-url https://download.pytorch.org/whl/cu121
pip install scikit-learn seaborn more-itertools matplotlib 
pip install tiktoken
\end{verbatim}

As always, we provide a Slurm batch submission script called \texttt{scholar.sh} in the skeleton code. You can use it to submit \emph{any} command lines to a GPU computing nodes. For instance, suppose the original command to run a python training script is \texttt{python main.py data}, then if run the following:
\begin{verbatim}
sbatch scholar.sh python main.py data
\end{verbatim}
Then the same job will be submitted to a GPU backend node. 

% Homework framework overview.
%
\subsubsection*{Skeleton}
%

%
\noindent Here is the folder structure that you should use:

% Skeleton figure.
%
\tikzstyle{every node}=[draw=black,thick,anchor=west]
\tikzstyle{selected}=[draw=red,fill=red!30]
\tikzstyle{core}=[draw=blue,fill=blue!30]
\tikzstyle{optional}=[dashed,draw=red,fill=gray!50]
%
\begin{tikzpicture}[%
    grow via three points={
        one child at (0.5,-0.7) and two children at (0.5,-0.7) and (0.5,-1.4)
    },
    edge from parent path={(\tikzparentnode.south) |- (\tikzchildnode.west)}
]
%
    \node {hw\homeworknumber\_transformer\_skeleton}
    child {node [selected] {my\_gpt.py}}
    child {node [selected] {generate.py}}
    child {node {input.txt}}
    child {node {scholar.sh}}
    child {node [selected] {README.md}};
\end{tikzpicture}

% Space.
%
\hfill

% Skeleton description.
%
\begin{itemize}
%
\item
    \textbf{{hw\homeworknumber\_transformer\_skeleton}:}
    %
    the top-level folder that contains all the files required in this homework.

\item
    \textbf{{my\_gpt.py}:}
    \underline{You will need to fill in this file}. The model script that defines the GPT Transformer model. 

\item
    \textbf{{generate.py}:}
    \underline{You will need to fill in this file}. The script that loades a trained model checkpoint and generates texts on new prompts.

\item
    \textbf{{input.txt}:}
    The training data. A text corpus from the book ``Twenty Thousand Leagues under the Sea.''

\item
    \textbf{README.md:}
    %
    \underline{You will need to fill in this file}. Your ReadMe should begin with a couple of \textbf{execution commands},
    e.g., ``python hw\homeworknumber.py data'', used to generate the outputs
    you report.
    %
    TA would replicate your results with the commands provided here.
    %
    More detailed options, usages and designs of your program can be followed.
    %
    You can also list any concerns that you think TA should know while running
    your program.
    %
    Note that put the information that you think it's more important at the
    top.
    %
    Moreover, the file should be written in pure text format that can be
    displayed with Linux ``less'' command.
    %
    You can refer to \texttt{interface.sh} for an example.
%

%
\end{itemize}
























\subsection*{Action Items:}

You should carefully look at the files \texttt{my\_gpt.py} and \texttt{generate.py}, where the necessary skeleton code has been provided for you. You should fill in your own code in the region marked by 

\begin{verbatim}
### TODO: You will need to complete the following
...
### END OF TODO ###
\end{verbatim}

In your report, include the answer of the following questions:
\begin{enumerate}

\item (2.0 pts)
It is important to first understand how an autoregressive language Transformer model is trained for the causal language modeling task. 

\begin{enumerate}

    \item Let $f_{\theta}$ be an autoregressive Transformer model which
    describes a conditional probability distribution $f_{\theta}(x_{t + 1} \mid
    x_1, x_2, \dots, x_t)$ over the $(t+1)$-th token $x_{t+1}$, given some input
    token sequence $[x_1, x_2, \dots, x_t]$. Now, given an input token sequence
    $X = [x_1, x_2, \dots, x_T]$, write down the probability that $f_{\theta}$
    would generates $X$, given the first starting token $x_1$.

    \begin{answerbox}[1.5in]
    The probability that the autoregressive Transformer model $f_{\theta}$ generates the token
    sequence $X = [x_1, x_2, \dots, x_T]$ given the first starting token $x_1$ can be expressed as:
    \begin{align*}
        P(X \mid x_1) = f_{\theta}(x_2 \mid x_1) \cdot f_{\theta}(x_3 \mid x_1, x_2) \cdots f_{\theta}(x_T \mid x_1, x_2, \dots, x_{T-1})
    \end{align*}

    This equation states that the probability of generating the entire sequence
    $X$ is the product of the conditional probabilities of each token given all
    previous tokens. The model generates each token in a sequential manner,
    conditioning on all previously generated tokens.
    
    \end{answerbox}

    \item The causal language modeling task asks the model to learn to predict
    the next tokens. Specifically, each training example is a token sequence
    $[x_1, x_2, \dots, x_T]$. We then ask the model to learn to predict the last
    token $x_T$ given $x_1, \dots, x_{T-1}$. If we denote the input token
    sequence by $X$, and the target token sequence by $Y$, what tokens are $X$
    and $Y$ composed of? Can you write down the mathematical expression for the
    Negative Log-Likelihood (NLL) loss given a dataset $\mathcal{D} = \{
    [x^{(i)}_1, \dots, x^{(i)}_{T_i}] \}_{i=1}^N$? You should expand and write
    the loss function equation using individual tokens, not $X$ or $Y$. \\
    \textbf{Hint:} You should take a look at the training loop in
    \texttt{my\_gpt.py}. Your answer should match how the loss is computed
    therein.

    \begin{answerbox}[2.5in]
    $X$ and $Y$ are composed of the same tokens, but $Y$ is shifted by one position
    to the right compared to $X$. This means that for each token in $X$, the
    corresponding token in $Y$ is the next token in the sequence. The NLL loss
    can be expressed as follows:
    \begin{align*}
        \mathcal{L}_{\text{NLL}}(\theta) = -\frac{1}{N} \sum_{i=1}^{N} \sum_{t=1}^{T_i} \log f_{\theta}(x^{(i)}_t \mid x^{(i)}_1, x^{(i)}_2, \dots, x^{(i)}_{t-1})
    \end{align*}
    where $N$ is the number of training examples, $T_i$ is the length of the
    $i$-th training example, and $f_{\theta}(x^{(i)}_t \mid x^{(i)}_1, x^{(i)}_2,
    \dots, x^{(i)}_{t-1})$ is the conditional probability of the $t$-th token
    given all previous tokens in the $i$-th training example. The loss is
    averaged over all training examples and all tokens in each example.

    % Students can write or type their answer here.
    \end{answerbox}

    \item Based on your answer to the previous question, during training, do we need to sample and generate model prediction tokens $\hat{y}$ and compare it with the ground-truth target tokens to compute the NLL loss? Explain why in either case in a few sentences.
    \begin{answerbox}[1.5in]
    During training, we do not need to sample and generate model prediction tokens $\hat{y}$.
    Instead, we compute the NLL loss directly using the model's output logits for
    each token in the sequence. The model outputs a probability distribution over
    the vocabulary for each token position, and we can compute the loss using the
    ground-truth target tokens. This approach allows us to efficiently compute the
    loss without the need for sampling, which would introduce additional
    randomness and make the training process less stable. By using the ground-truth
    target tokens, we can directly optimize the model's parameters to minimize
    the NLL loss, leading to more effective training.

    % Students can write or type their answer here.
    \end{answerbox}

    \item Now, let $\mathbf{D} \in \mathbb{N}^{B \times T}$ be the token indices matrix for a batch of input text, where $B$ is the batch size and $T$ is the total sequence length (for simplicity, we will assume, both here and in the code, that every training example have the same token sequence length, thus avoiding padding). Explain how to obtain the input token indices matrix $\mathbf{X}$ and the target token indices matrix $\mathbf{Y}$ from $\mathbf{D}$. What are the shapes of both matrices? \\
    You should fill in the missing parts in the \texttt{TextDataset} class in \texttt{my\_gpt.py} file. You answer to this question should exactly translate to your code.\\
    \textbf{Hint:} Pay attention that the Transformer model output at position $t$ is actually meant for and aligned to the $(t+1)$-th token in the sequence. This is known as the ``shift-by-one'' problem.
    
    \begin{answerbox}[2.5in]
    The input token indices matrix $\mathbf{X}$ is obtained by taking all columns of $\mathbf{D}$ except the last one. This can be expressed as:
    \[
    \mathbf{X} = \mathbf{D}[:, :-1]
    \]
    The shape of $\mathbf{X}$ is $(B, T-1)$, where $B$ is the batch size and $T$ is the total sequence length.

    The target token indices matrix $\mathbf{Y}$ is obtained by taking all columns of $\mathbf{D}$ except the first one, and shifting the sequence to the left. This can be expressed as:
    \[
    \mathbf{Y} = \mathbf{D}[:, 1:]
    \]
    The shape of $\mathbf{Y}$ is also $(B, T-1)$, where $B$ is the batch size and $T$ is the total sequence length. The target tokens in $\mathbf{Y}$ are aligned with the input tokens in $\mathbf{X}$, such that the model's output at position $t$ corresponds to the target token at position $t+1$.

    % Students can write or type their answer here.
    \end{answerbox}
    
\end{enumerate}

\item (0.5 pt)
Our Transformer model will use the Rotary Positional Encoding (RoPE), one of the state-of-the-art PE for LLMs. You should fill in the missing parts of the \texttt{RotaryEmbedding} class and \texttt{apply\_rotary\_emb()} method in \texttt{my\_gpt.py}. \\
In Q2, we have written out the expressions $\text{ROPE}_{2i}(\text{pos})$ and $\text{ROPE}_{2i + 1}(\text{pos})$, which are the ROPE values for positional indices $\text{pos}$.
 Write down the RoPE rotation matrix $\mathbf{R}_{\text{pos}} \in \mathbb{R}^{d_{\text{head}} \times d_{\text{head}}}$, where $d_{\text{head}}$ is the dimension of the query/key states for each attention head. Then, let $m$ and $n$ be two distinct position indices. Let $\mathbf{q}_m, \mathbf{k}_n \in \mathbf{R}^{d_{\text{head}}}$ be the query and key state for the tokens at $m$-th and $n$-th position (after query/key projection), respectively. Write down the expression for the attention logit $\mathbf{q}_m^T \mathbf{k}_n$ in which the rotations $\mathbf{R}$ takes part. Explain why RoPE is able to model the ``relative'' position information of the $m$-th and $n$-th token.

\begin{answerbox}[2.8in]
 The RoPE rotation matrix $\mathbf{R}_{\text{pos}}$ is defined as:
\begin{align*}
    \mathbf{R}_{\text{pos}} = \begin{bmatrix}
        \cos(\text{ROPE}_{2i}(\text{pos})) & -\sin(\text{ROPE}_{2i}(\text{pos})) \\
        \sin(\text{ROPE}_{2i}(\text{pos})) & \cos(\text{ROPE}_{2i}(\text{pos}))
    \end{bmatrix}
\end{align*}
The attention logit $\mathbf{q}_m^T \mathbf{k}_n$ can be expressed as:
\begin{align*}
    \mathbf{q}_m^T \mathbf{k}_n &= \left(\mathbf{R}_{\text{pos}}(\text{pos}_m) \cdot \mathbf{q}_m\right)^T \cdot \left(\mathbf{R}_{\text{pos}}(\text{pos}_n) \cdot \mathbf{k}_n\right) \\
    &= \mathbf{q}_m^T \cdot \mathbf{R}_{\text{pos}}(\text{pos}_m)^T \cdot \mathbf{R}_{\text{pos}}(\text{pos}_n) \cdot \mathbf{k}_n
\end{align*}

The RoPE is able to model the relative position information of the $m$-th and
$n$-th token because the rotation matrices
$\mathbf{R}_{\text{pos}}(\text{pos}_m)$ and
$\mathbf{R}_{\text{pos}}(\text{pos}_n)$ depend on the positional indices $m$ and
$n$. The relative position information is captured in the product of the two
rotation matrices, which encodes the angular difference between the two
positions. This allows the model to learn and utilize the relative positional
information during attention computation.

    % Students can write or type their answer here.
    \end{answerbox}

    \item (0.5 pt) Now, you should fill in the missing parts of the
    \texttt{CausalSelfAttention} class in \texttt{my\_gpt.py}, which is the
    attention computation. You will need to implement the forward pass of the
    attention module, including the query/key/value projections, and the
    attention score computation. You can use the PyTorch function
    \texttt{torch.einsum()} to compute the attention scores.

 \begin{answerbox}[2in]
    The attention score computation is given by the following equation:
    \begin{align*}
        \text{attn\_scores} = \frac{\mathbf{Q} \cdot \mathbf{K}^T}{\sqrt{d_{\text{head}}}}
    \end{align*}
    where $\mathbf{Q} \in \mathbb{R}^{B \times H \times T \times d_{\text{head}}}$ is the query matrix, $\mathbf{K} \in \mathbb{R}^{B \times H \times T \times d_{\text{head}}}$ is the key matrix, and $d_{\text{head}}$ is the dimension of each attention head. The attention scores are then passed through a softmax function to obtain the attention weights.
    \begin{align*}
        \text{attn\_weights} = \text{softmax}(\text{attn\_scores}, \text{dim}=-1)
    \end{align*}

    % Students can write or type their answer here.
    \end{answerbox}


\item (0.5 pts)
Now, you should fill in the missing parts of \texttt{CausalSelfAttention.forward()} method in \texttt{my\_gpt.py}, which is the attention computation. Pay attention when computing and applying the causal mask, since we want to ensure every query state can only attend to key states that appear before (i.e. having a positional index smaller than) said query state. \\
You are now ready to train the Transformer! Train your model with the default parameters (batch size 32, block size 128, train split 0.9, \& num epochs 4) with command \\
\texttt{python my\_gpt.py}
(or if you are Slurm sbatch, run with \texttt{sbatch scholar.sh python my\_gpt.py}). \\
Report the validation loss at each epoch, and the training loss at the final step from the console output (or if you are using Slurm sbatch, the console logs will be saved to text files under the \texttt{slurm\_logs/} directory.)
\begin{answerbox}[2in]
    The validation loss at each epoch is as follows:
    \begin{itemize}
        \item Epoch [1/4] Validation Loss (per sample): 0.0038
        \item Epoch [2/4] Validation Loss (per sample): 0.0031
        \item Epoch [3/4] Validation Loss (per sample): 0.0027
        \item Epoch [4/4] Validation Loss (per sample): 0.0026

    \end{itemize}
    The training loss at the final step is 0.0031.
    % Students can write or type their answer here.
    \end{answerbox}

\item (1.0 pt)
Once the training is finished and we have a model checkpoint, we can use it to generate new texts. You will need to fill in the \texttt{generate()} method in \texttt{generate.py}.  (You may ignore the parts for when KV cache is involved, and come back later).
\begin{enumerate}
    \item 

    Describe what do the arguments \texttt{temperature} and \texttt{topk} do,
    and what is the qualitative effect when their values are changed.

    \begin{answerbox}[1.5in]
    % Students can write or type their answer here.

    The temperature parameter scales the logits before applying the softmax.
    Lower values (e.g., 0.5) make the distribution sharper, leading to more
    deterministic and conservative choices. Higher values (e.g., 1.5) flatten
    the distribution, making sampling more random and diverse.  

    The top-k parameter limits the number of tokens considered for sampling to
    the top-k most probable ones. A smaller top-k (e.g., 5) results in less
    diversity and more focused generation, while a larger top-k (e.g., 40)
    allows for more varied outputs.

    \end{answerbox}

    \item Generate two texts with the initial prompt ``\texttt{The ship was}", one
    with temperature 1.0, top-k 40, and another one with temperature 0.1, top-k
    5. Report the your generated texts in your report PDF.

    \begin{answerbox}[1.5in]
        \begin{itemize}
            \item \textbf{temperature 1.0, top-k 40:} The ship was, he
            approached the depth of New Zealand; put into Calcutta, 7th April,
            1828, and returned to France, where he was warmly welcomed by
            Charles X.
            \item \textbf{temperature 0.1, top-k 5:}
            The ship was steered entirely by the compass and the log.
        \end{itemize}

    % Students can write or type their answer here.
    \end{answerbox}
    
\end{enumerate}

\item (1.5 pts)
For autoregressive Transformer models, generating new texts given some input text $X$ requires feeding $X$ again and again to the model for each new token being generated. This incurs a lot of redundant computations and slows down the model generation speed. Thus, in modern LLMs, we leverage a technique known as \textbf{Key Value Caching (KV Caching)}, to cache the reusable intermediate representations.
\begin{enumerate}

    \item Given two input sequence, $X = [x_1, \dots, x_T]$ and $X' = [x_1,
    \dots, x_T, x_{T+1}]$, where $x_{T+1}$ can be considered as the newly
    generated token. Denote $\mathbf{H}^{(l)} = [\mathbf{h}_l^{(l)}, \dots,
    \mathbf{h}_T^{(l)}] \in \mathbb{R}^{T \times d}, \mathbf{H'} =
    [\mathbf{h}_1^{(l)}, \dots, \mathbf{h}_{T+1}^{(l)}] \in \mathbb{R}^{(T+1)
    \times d}$ the output hidden states of the $l$-th layer (with
    $\mathbf{H}^{(0)}$ and $\mathbf{H'}^{(0)}$ being the input hidden states to
    the first layer). Denote $\mathbf{W}_Q, \mathbf{W}_K, \mathbf{W}_V,
    \mathbf{W}_O$ the query, key, value, and output projection matrices,
    respectively. Describe, in the attention computation, which intermediate
    states are being recomputed between $X$ and $X'$, and could have been saved
    in a cache to avoid recomputation. Also, describe how using or not using
    causal masking affects what states might be cached.

    \begin{answerbox}[1.5in]

    Given two sequences $X = [x_1, \dots, x_T]$ and $X' = [x_1, \dots, x_T,
    x_{T+1}]$, the attention computation for $x_{T+1}$ requires computing a new
    query $Q_{T+1}$, and attending to keys and values $K_1, \dots, K_{T+1}$ and
    $V_1, \dots, V_{T+1}$. The projections $K_1, \dots, K_T$ and $V_1, \dots,
    V_T$ are shared with $X$ and can be cached to avoid recomputation. If causal
    masking is used, past tokens do not attend to future ones, so the hidden
    states for $x_1, \dots, x_T$ remain unchanged and caching is valid. Without
    causal masking, earlier tokens can attend to $x_{T+1}$, so their hidden
    states must be recomputed, making caching less effective.

    \end{answerbox}

    \item It is very important that you assign the correct RoPE when working
    with KV cache. Take the input $X' = [x_1, \dots, x_T, x_{T+1}]$ as an
    example, and suppose the model has seen the first $T$ tokens, $[x_1, \dots,
    x_T]$, before and has saved the relevant cache, while $x_{T+1}$ is the
    ``new'' token in this forward pass. Describe which RoPE rotations (more
    specifically, RoPE with which range/sequence of positional indices) should
    be applied to the query and key states. In addition, describe what causal
    mask should be used here and applied to the attention scores with KV caching
    enabled. 

    \begin{answerbox}[1.5in]

        With KV caching enabled, the RoPE rotations for the query and key states
        should be applied only to the new token $x_{T+1}$, using the RoPE
        rotation for the position $T+1$, while the key states for $x_1, \dots,
        x_T$ are cached, and should not be re-rotated. 

        The causal mask should be applied to ensure that the attention scores
        for $x_{T+1}$ only attend to the previous tokens $x_1, \dots, x_T$,
        which is trivially satisfied in this case when we are only computing the
        attention score for $x_{T+1}$.

    % Students can write or type their answer here.
    \end{answerbox}
    
    \item You should now be ready to complete the KV cache implementation. Fill
    in the relevant parts in \texttt{apply\_rotary\_emb()},
    \texttt{CausalSelfAttention.forward()}, and \texttt{generate()}. Run the
    generation again with KV-caching enabled, via \texttt{python generate.py
    --use\_kv\_cache}, with the same two configurations of temperature and top-k
    we used in the previous question (one with temperature 1.0, top-k 40, and
    another one with temperature 0.1, top-k 5). Report your generated text. Are
    they similar to what you generated previously without KV caching? \\
    \textbf{Hint}: One debugging trick to ensure consistent generation with and
    without KV caching, is to set top-k to 1. This is effectively greedy
    sampling.

    \begin{answerbox}[4in]

    The ship was called by the circular storm of the southern hemisphere. Ah,
    that Gulf Stream! It deserves its name of the King of Tempests. It is that
    which causes those formidable cyclones, by the difference of temperature
    between its air and its currents. A shower of fire had succeeded the rain.
    The drops of water were changed to sharp spikes. One would have thought that
    Captain Nemo was courting a death worthy of himself, a death by lightning.
    As the \_Nautilus\_, pitching dreadfully, raised its steel spur in the air, it
    seemed to act as a conductor, and I saw long sparks burst from it. Crushed
    and without strength I crawled to the panel, opened it, and descended to the
    saloon. The storm was then at its height. It was impossible to stand upright
    in the interior of the \_Nautilus\_. Captain Nemo came down about twelve. I
    heard the reservoirs filling by degrees, and the \_Nautilus\_ sank slowly
    beneath the waves. Through the open windows in the saloon I saw large fish
    terrified, passing like phantoms in the water. Some were struck before my
    eyes. The \_Nautilus\_ was still descending. I thought that at about eight
    fathoms deep we should find a calm. But no! the upper beds were too
    violently agitated for that. We had to seek repose at more than twenty-five
    fathoms in the bowels of the deep. But there, what quiet, what silence, what
    peace! Who could have told that such a hurricane had been let loose on the
    surface of that ocean?

    The text is the same with and without KV caching.

    \end{answerbox} 
    
\end{enumerate}



\end{enumerate}




%############################################################
\newpage
\subsection*{Submission Instructions}
%
Please read the instructions carefully.
%
Failed to follow any part might incur some score deductions.

% Space.
%
\hfill

% PDF.
%
\subsection*{\bf PDF upload}
%
The report PDF must be uploaded on Gradescope.

% Space.
%
\hfill

% Code.
%
\subsection*{\bf Code upload}
%
\noindent \textbf{Naming convention}:
%
[firstname]\_[lastname]\_hw\homeworknumber
\\
%
All your submitting code files, a ReadMe, should be included in one folder. Create a folder [firstname]\_[lastname]\_hw\homeworknumber \ and put both the implemented skeleton folders inside it. Then tar it.\\
%
The folder should be named with the above naming convention.
%
For example, if my first name is ``Bruno'' and my last name is ``Ribeiro'',
then for Homework \homeworknumber, my file name should be
``bruno\_ribeiro\_hw\homeworknumber''.

% Space.
%
\hfill

% Pack your submission.
%
\noindent \textbf{Tar your folder:}
%
[firstname]\_[lastname]\_hw\homeworknumber.tar.gz
\\
%
Remove any unnecessary files in your folder, such as training datasets.
%
Make sure your folder structured as the tree shown in Overview section.
%
Compress your folder with the the command:
%
\textbf{
    tar -czvf bruno\_ribeiro\_hw\homeworknumber.tar.gz bruno\_ribeiro\_hw\homeworknumber
.}

% Space.
%
\hfill

% Submit.
%
\noindent \textbf{Submit:}
%
{\bf Brightspace}
\\
%
Please submit your compressed file on \textbf{Brightspace}.
%
Please make sure you didn't use any library/source explicitly forbidden to use.
%
If such library/source code is used, you will get 0 pt for the coding part of
the assignment.
%
If your code doesn't run on scholar.rcac.purdue.edu, then even if it compiles
in another computer, your code will still be considered not-running and the
respective part of the assignment will receive 0 pt.


\end{document}
