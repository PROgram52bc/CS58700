% =============================================================================
% *****************************************************************************
% -----------------------------------------------------------------------------
% ## Table Of Contents
%
% - Table Of Contents
% - Document Definition
% - Package
% - Beginning Of Document
% - Theoretical Questions
% - Q1.1
% - Q1.2
% - Q1.3
% - Q1.4
% - Programming
% - Q2
% - Q3
% - Submission Instruction
% - End Of Document
% -----------------------------------------------------------------------------
% *****************************************************************************
% =============================================================================


% =============================================================================
% *****************************************************************************
% -----------------------------------------------------------------------------
% ## Document Definition
% -----------------------------------------------------------------------------
% *****************************************************************************
% =============================================================================


% Homework template.
% NOTE:
% Be sure to define your team members with the \team command.
% Be sure to define the problem set with the \ps command.
% Be sure to use the \answer command for each of your answers.
%
\documentclass{article}


% =============================================================================
% *****************************************************************************
% -----------------------------------------------------------------------------
% ## Package
% -----------------------------------------------------------------------------
% *****************************************************************************
% =============================================================================


% Package.
%
\usepackage[utf8]{inputenc}
\usepackage{graphicx}
\usepackage{amssymb}
\usepackage{amsmath}
\usepackage{epstopdf}
\usepackage{enumerate}
\usepackage{booktabs}
\usepackage{multirow}
\usepackage[usenames,dvipsnames,svgnames,table]{xcolor}
\usepackage{url}
\usepackage{dsfont}
\usepackage{tikz}
\usetikzlibrary{trees}
\usepackage{makecell}
\usepackage{hyperref}
\usepackage{listings}
\usepackage{bbm}
\usepackage{bm}
\usepackage{xspace}
\usepackage{xcolor}
\usepackage{algorithm}
\usepackage{algpseudocode}
\usepackage{cleveref}
\usetikzlibrary{arrows,positioning,shapes.multipart}

\graphicspath{{./hw3/}}
% New commands.
%
\newcommand{\class}{ CS58700-DPL Spring 2025 }
\newcommand{\website}{
    {\tiny\url{https://www.cs.purdue.edu/homes/ribeirob/courses/Spring2025}}
}
\newcommand{\homeworknumber}{3}
\newcommand{\duedate}{  {\bf 6:00pm}, Wednesday, March 26th (open until 4:00am next day)} 
\newcommand{\code}{CODE}

% New commands.
%
\setlength{\parskip}{1pc}
\setlength{\parindent}{0pt}
\setlength{\topmargin}{-1pc}
\setlength{\textheight}{8.5in}
\setlength{\oddsidemargin}{0pc}
\setlength{\evensidemargin}{0pc}
\setlength{\textwidth}{6.5in}

% New commands.
%
\newcommand{\var}{\mbox{var}}
\newcommand{\cov}{\mbox{cov}}

% New commands.
%
\newcommand{\mR}{\bm{R}}
\newcommand{\mA}{\bm{A}}
\newcommand{\mH}{\bm{H}}
\newcommand{\mW}{\bm{W}}
\newcommand{\mX}{\bm{X}}
\newcommand{\mY}{\bm{Y}}

% New commands.
%
\newcommand{\mPhi}{\bm{\Phi}}
\newcommand{\vb}{\bm{b}}
\newcommand{\vc}{\bm{c}}
\newcommand{\vh}{\bm{h}}
\newcommand{\vm}{\bm{m}}
\newcommand{\vx}{\bm{x}}

% New commands.
%
\newcommand*\dbar[1]{\overline{\overline{\lower0.2ex\hbox{$#1$}}}}
\newcommand{\harrow}[1]{
    \mathstrut\mkern2.5mu#1\mkern-11mu\raise1.6ex\hbox{
        $\scriptscriptstyle\rightharpoonup$
    }
}

% New commands.
%
\newcommand{\newpart}{
    \stepcounter{partno}
    \noindent
    {\bf (\alph{partno})}
}

% New commands.
%
\newcommand{\header}{
    \newpage
    \noindent
    \framebox{
        \vbox{
            \class Homework \hfill --- Homework \homeworknumber --- \hfill Last
            update: \today
            \\
            \website \hfill {\color{red} Due: \duedate}
        }
    }
    \bigskip
    \newline
    %
    {\bf Instructions and Policy:}
    %
    Each student should write up their own solutions independently, no copying
    of any form is allowed.
    %
    You MUST to indicate the names of the people you discussed a problem with;
    ideally you should discuss with no more than two other people.
    \\
    {\color{red} YOU MUST INCLUDE YOUR NAME IN THE HOMEWORK.}
    \\
    You need to submit your answer in PDF.
    %
    {\LaTeX} is typesetting is encouraged but not required.
    %
    Please write clearly and concisely - clarity and brevity will be rewarded.
    %
    Refer to known facts as necessary.
    \newline
}

% New commands.
%
\newcounter{questionno}
\setcounter{questionno}{-1}
\newcounter{partno}

% New commands.
%
\newcommand{\question}[1]{
    \noindent
    \newline
    \stepcounter{questionno}
    \setcounter{partno}{0}
    {\bf Q\arabic{questionno} (#1 pts):}
}


% =============================================================================
% *****************************************************************************
% -----------------------------------------------------------------------------
% ## Beginning Of Document
% -----------------------------------------------------------------------------
% *****************************************************************************
% =============================================================================


% Beginning of document.
%
\begin{document}

% Common homework head context.
%
\header
\question{%
    {\color{red}0pts correct answer, -1,000pts incorrect answer: (0,-1,000)}%
}
%
A correct answer to the following questions is worth 0pts.
%
An incorrect answer is worth -1,000pts, which carries over to other homeworks
and exams, and can result in an F grade in the course.
%
\begin{enumerate}[(1)]
\item Student interaction with other students / individuals:
\begin{enumerate}[(a)]
\item I have copied part of my homework from another student or another person (plagiarism).
\item \textbf{Yes, I discussed the homework with another person but came up with my own answers. Their name(s) is (are): Jason Fotso}
\item No, I did not discuss the homework with anyone
\end{enumerate}
\item On using online resources:
\begin{enumerate}[(a)]
\item I have copied one of my answers directly from a website (plagiarism).
\item \textbf{I have used online resources to help me answer this question, but I came up with my own answers (you are allowed to use online resources as long as the answer is your own). Here is a list of the websites I have used in this homework:\\ \url{https://pytorch.org/docs/main}}
\item I have not used any online resources except the ones provided in the course website.
\end{enumerate}

\end{enumerate}


% =============================================================================
% *****************************************************************************
% -----------------------------------------------------------------------------
% ## Theoretical Question
% -----------------------------------------------------------------------------
% *****************************************************************************
% =============================================================================


% Start from new page.
%
\newpage

% Space.
%
\hfill

% Homework as a section.
%
\setcounter{section}{\homeworknumber}
\section*{%
    Homework \homeworknumber: Graph Convolutions and Graph Neural Networks
}

% Learning object.
%
\noindent \textbf{Learning Objectives:}
%
Let students understand basic concepts that are used to add inductive biases and invariances in graph problems.
% Space.
%
\hfill

% Learning Outcomes.
%
\noindent \textbf{Learning Outcomes:}
%
After finishing this homework, students will be able to create new inductive
biases to solve new graph-related machine learningtasks.

\section{Q1: Conceptual Part {\bf (2 pts + (bonus) 1 pt)}}

Please answer the following questions \textbf{concisely}. All the answers, along with your name and email, should be clearly typed in some editing software, such as Latex or MS Word.

\begin{enumerate}


\item {\bf (0.4 pt)} Prove mathematically why CNNs are sensitive to image rotations. \\
{\bf Hint:} Think about how the combination of kernels over small image patches induces sensitivity.

Because for each kernel 

$$O(i, j) = \sum_m \sum_n I(i+m, j+n) \cdot K(m, n)$$,

where $I$ is the input image, $K$ is the kernel, and $O$ is the output image,
the convolutional layer is sensitive to the order of the input image, namely,
$K(m, n) \neq K(m', n')$, where $m'$ and $n'$ are the rotated coordinates of $m$
and $n$. Therefore, the convolutional layer is sensitive to the order of the
input image, and the output of the convolutional layer changes when the input
image is rotated.

Additionally, the vectorization of the image patches is also order-sensitive,
which makes the network sensitive to the rotation of the input image. In order
to make the model invariant to rotation, one has to use techniques such as data
augmentation or G-invariant CNNs.

\vspace{3.5in}

\item {\bf (0.4 pt)} Early stopping uses the validation data to decide which parameters we should keep during our SGD optimization. Explain why models obtained by early stopping tend to generalize better than models obtained by running a fixed number of epochs.
Also explain why early stopping should never use the training data.

Early stopping tracks the performance of the model on the validation set, which
is independent from the training set, and is a reliable metric to measure the
generalization performance of the model. When the model starts to overfit the
training data, the performance on the validation set will start to decrease.  By
stopping the training when the model starts to overfit the training data, we can
prevent the model from memorizing the training data and improve the
generalization performance. 

If we use the training data to decide when to stop, it will not help the model
detect overfitting, because the model will always perform better on the training
data as it is trained on it. Therefore, the model will overfit the training data
and generalize poorly to unseen data.

\vspace{3.5in}


\item {\bf (0.4 pt)} Propose a Metropolis-Hastings algorithm designed to eliminate watermarks and blur from images. Specifically, assume that we have access to the weights of a pre-trained neural network $f: [0, 1]^{d \times d \times 3} \to [0, 1]^{d \times d \times 3}$ that was trained using the L2 loss to intelligently introduce watermarks or blurs to input raw images (assume all images have 3 channels (RGB) and channel values normalized to range $[0, 1]$). Now, given a single watermarked or blurred image $\mathbf{Y} \in [0, 1]^{d \times d \times 3}$, give the pseudocode for a Metropolis-Hastings algorithm that generates $N$ images $\{\mathbf{X}_i\}_{i=1}^N$ using $f$ that are versions of $\mathbf{Y}$ with the watermarks removed and unblurred. \\
{\bf Hint:} Since the model $f$ used L2 loss to train, what is its probabilistic assumption (e.g. what is the noise distribution)? How do you express the data likelihood $P(\mathbf{Y} \mid \mathbf{X}; f)$ under this probabilistic assumption? \\
{\bf Hint:} How to initialize $\mathbf{X}$? Starting from a random tensor uniformly sampled from $[0, 1]^{d \times d \times 3}$ would be very inefficient because it would take the algorithm a long time to reach the ``good'' region. What is a better way to initialize?

Start from a random image $\mathbf{X}$, initialized to be $\mathbf{Y}$, and for
each iteration, propose a new image $\mathbf{X}'$ by adding a small amount of
noise to $\mathbf{X}$. Calculate the acceptance probability $A(\mathbf{X},
\mathbf{X}')$ using the Metropolis-Hastings acceptance probability formula: $A =
\min\left(1, \frac{P(\mathbf{X}')}{P(\mathbf{X})}\right)$. Here, $P(\mathbf{X})$
is the likelihood of the noisy image $\mathbf{Y}$, under the assumption that
$\mathbf{X}$ is the clean image. Since the model $f$ was trained using L2 loss,
the noise distribution is Gaussian. Therefore, $P(\mathbf{Y}|\mathbf{X}; f)$ is
the Gaussian likelihood of the noisy image $\mathbf{Y}$ given the clean image
$\mathbf{X}$.

Generate a random number between 0 and 1, and if it is less than $A$, accept
$\mathbf{X}'$ as the new image. Repeat this process for $N$ iterations to obtain
$N$ images $\{\mathbf{X}_i\}_{i=1}^N$.

\vspace{3.5in}


\item {\bf (0.8 pt + (bonus) 1 pt)} Consider a one hidden-layer Multi Layer Perceptron with ReLU as activation without biases (for simplicity). The output $\hat{y}\in \mathbb{R}$ of this network for an input $x \in \mathbb{R}^d$ can be defined as $\hat{y} = f(x)$, where $f(x; W_1,W_2) := ReLU(W_2^T  ReLU( W_1^T x))$, where $W_1\in \mathbb{R}^{d\times d_1},W_2\in \mathbb{R}^{d_1\times 1}$ are arbitrary weight matrices we will learn from data.
\begin{enumerate}
    \item Prove that $\forall x \in \mathbb{R}^d$ and $\forall \alpha \in \mathbb{R}^+$ (positive real numbers), we have $f(x; W_1,W_2) = f(x; \alpha W_1,W_2/\alpha)$.

    Because the ReLU function can be splitted into two segments $ReLU(x) = x$ if
    $x > 0$ and $ReLU(x) = 0$ if $x \leq 0$, we prove separately that for the two segments,
    multiplication by a positive constant can distribute into/out of the ReLU function.
    
    Namely, for $x > 0$, $ReLU(\alpha x) = \alpha x = \alpha ReLU(x)$, and for
    $x \leq 0$, $ReLU(\alpha x) = 0 = \alpha ReLU(x)$. Therefore, $ReLU(\alpha
    x) = \alpha ReLU(x)$ for all $x \in \mathbb{R}$ and $\alpha \in
    \mathbb{R}^+$.

    Therefore, we have $f(x; W_1, W_2) = ReLU(W_2^T ReLU(W_1^T x)) = ReLU(W_2^T
    \alpha / \alpha ReLU(W_1^T x)) = ReLU(W_2^T\alpha ReLU(\alpha W_1^T x) =
    f(x; \alpha W_1,W_2/\alpha)$

    \vspace{1.5in}

    \item  
  Consider the same setting as above. Let the negative log-likelihood be $L(W_1, W_2, x, y) = (f(x; \\W_1,W_2) - y)^2$ (hence, we are assuming a model with additive standard Gaussian noise as error). Assume $(W^*_1,W^*_2)$ is a critical point for the negative log-likelihood, i.e., $\frac{\partial L}{\partial W_1}|_{(W_1,W_2) = (W^*_1,W^*_2)}=0$ and $\frac{\partial L}{\partial W_2}|_{(W_1,W_2) = (W^*_1,W^*_2)}=0$, where $W^{\alpha,*}_1=\alpha W^*_1$, $W^{\alpha,*}_2=W^*_2/\alpha$. Prove that $\forall \alpha \in \mathbb{R}^+$, $(W^{\alpha,*}_1, W^{\alpha,*}_2)$ is also a critical point. \\
  \textbf{Warning}: The derivative of a ReLU function at point $0$ is undefined. For simplicity, you do not need to consider this edge case.

  Since from the previous setting, we have $f(x; W_1, W_2) = f(x; \alpha W_1,
  W_2/\alpha)$, we have $L(W_1, W_2, x, y) = L(\alpha W_1, W_2/\alpha, x, y)$,
  for the given loss function. Therefore, the critical points of the loss
  function are the same for the two settings. Since $(W^*_1, W^*_2)$ is a
  critical point, for all $x$ such that $ReLU(W_1^T x) \neq 0$ and $ReLU(W_2^T
  ReLU(W_1^T x)) \neq 0$, we have 
  
  $$\frac{\partial L}{\partial W_1}|_{(W_1,W_2) = (W^*_1,W^*_2)} =
  \frac{\partial L}{\partial W_1}|_{(W_1,W_2) = (W^{\alpha,*}_1,W^{\alpha,*}_2)}
  = 0$$ and $$\frac{\partial L}{\partial W_2}|_{(W_1,W_2) = (W^*_1,W^*_2)} =
  \frac{\partial L}{\partial W_2}|_{(W_1,W_2) = (W^{\alpha,*}_1,W^{\alpha,*}_2)}
  = 0$$. Therefore, $(W^{\alpha,*}_1, W^{\alpha,*}_2)$ is also a critical point.
  
    \newpage
    
    \item {\bf (bonus 0.5 pt)}
    Using the same setting as (b), assume $(W_1^*,W_2^*)$ is a local minima of the negative log-likelihood. We can show $(W_1^{\alpha,*},W_2^{\alpha,*})$ is also a local minimum point (no need to prove this property), where $W^{\alpha,*}_1=\alpha W^*_1$, $W^{\alpha,*}_2=W^*_2/\alpha$. We define a sharp function using the Hessian's Frobenius norm $$\text{sharp}(W_1,W_2) := \left\Vert \begin{bmatrix} \nabla^2_{W_1}(L(W_1,W_2,x,y))  & \mathbf{0}\\
    \mathbf{0} & \nabla^2_{W_2}(L(W_1,W_2,x,y))
    \end{bmatrix}\right\Vert_F,$$ which can be used as a proxy measure of the sharpness of the likelihood landscape near the local minimum of the negative log-likelihood (since the Hessian $\nabla^2_{W_h}(L(W_1,W_2,x,y))$, $h \in \{1,2\}$, at a local minimum is usually positive definite, it measures the local curvature).

    Function $f(x;W_1,W_2)$ is said to have a sharper loss at a local minimum $(W_1^*,W_2^*)$ than another local minimum $(W_1^{**},W_2^{**})$ if 
    $$\text{sharp}(W^*_1,W^*_2) > \text{sharp}(W^{**}_1,W^{**}_2).$$
    Show that, for an arbitrary local minima $(W_1,W_2)$, there is an arbitrary constant $\alpha > 0$ such that the sharpness of the loss function increases, i.e., $$\text{sharp}(W^{\alpha,*}_1,W^{\alpha,*}_2) > \text{sharp}(W^{*}_1,W^{*}_2).$$

    
    \textbf{Hint 1:} Calculate the relationship between $\nabla^2_{W_1}(L(W_1,W_2,x,y))|_{(W_1,W_2) = (W^{\alpha,*}_1,W^{\alpha,*}_2)}$ and\\ $\nabla^2_{W_1}(L(W_1,W_2,x,y))|_{(W_1,W_2) = (W^{*}_1,W^{*}_2)}$ (also for $\nabla^2_{W_2}$). Both of them are (assumed to be) positive definite, which means positive eigenvalues. To calculate the sharpness, we can then use the property that trace of a matrix is equal to the sum of its eigenvalues to determine if there exists positive elements in the diagonal. From the definition of the Frobenius norm, we can easily see the norm is bounded below by the sum of any subset of positive element in the matrix. Finally, we can show how the sharpness can be increased by adjusting the positive element (with $\alpha$) in the matrix to increase the Frobenius norm. \\
    \textbf{Hint 2}
    {Frobenius norm of a $m \times n$ matrix $\mathbf{A}$ is defined as, 
        \begin{displaymath}
           \lvert\lvert\mathbf{A}\rvert\rvert_{F} = \sqrt{\sum_{i=1}^{m}\sum_{j=1}^{n}\lvert a_{ij}\rvert^2} .
        \end{displaymath}
        }

    \vspace{3in}
    

    \newpage
    \item {\bf (bonus 0.5 pt)} This question will explain why covariance matrix of Metropolis-Hasting proposals are very important. Given the training data $\{\mathbf{x}_i, \mathbf{y}_i\}_{i=1}^{n}$ for a supervised learning model task $p(W_1, W_2| \{\mathbf{x}_i, \mathbf{y}_i\}_{i=1}^{n})$, where $W_1, W_2$ are the paremeters of the model $f(x;W_1,W_2)$ stated at the beginning of this question,  we would like to apply the following Bayesian averaging procedure $$p_{Bayesian}(\mathbf{y}|\mathbf{x}; \{\mathbf{x}_i, \mathbf{y}_i\}_{i=1}^{n}) = \frac{1}{K}\sum_{i=1}^{k}p(\mathbf{y}|\mathbf{x};(W^{(k)}_1,W^{(k)}_2))$$ using a Metropolis-Hastings (MH) procedure to obtain $K$ independent samples of the posterior as follows
    $$(W^{(k)}_1, W^{(k)}_2) \sim P(W_1, W_2 | \{\mathbf{x}_i, \mathbf{y}_i\}_{i=1}^{n} ), \qquad k \in{1, 2, \ldots, K}.$$ 
    
    Assume we have calculated the rejection rate $\gamma = \frac{\text{Number of times the samples were rejected}}{\text{Number of sampled MH proposals}}$. Now assume the Metropolis-Hastings procedure has a sampling proposal with covariance matrix $\mathbf{I}$, i.e., $q(W_{1,t+1}|W_{1,t}) \sim \text{Normal}(W_{1,t},\mathbf{I} )$. Consider $W_1, W_2$ as vectors, assume we know $(W_1^*, W_2^*)$ is a local minimum point as in 4(b), we further assume the initial state for the $k$-th MH sampling procedure is defined as $(W^{(k)}_{1,0}, W^{(k)}_{2,0}) = (W_1^{\alpha^{(k)},*}, W_2^{\alpha^{(k)},*})$, $\alpha^{(1)},...,\alpha^{(K)}\in \mathbb{R}^+$. Now assume $\alpha^{(1)}=1$, you observe there exists $k\in {1,...,K}$, such that the $k$-th Metropolis-Hastings procedure has much higher rejection rate than the $1$-st Metropolis-Hastings procedure. Can you give one possible explanation for this phenomenon?
    \\
    {\bf Hint 1:} Use the insights from Q4(b) about the sharpness of the likelihood at different local minimum points.\\
    {\bf Hint 2:} A 2D drawing of the phenomenon might help you explain it.\\
    
    
     
\end{enumerate}


\end{enumerate}



% =============================================================================
% *****************************************************************************
% -----------------------------------------------------------------------------
% ## Programming
% -----------------------------------------------------------------------------
% *****************************************************************************
% =============================================================================


% Start from new page.
%
\newpage

% Beginning of programming.
%
\section*{Programming Part {\bf (8 pts)}}

In this part, you are going to perform multiple tasks: (1) HMC sampler for MLP parameters, and (2) CNN and G-Invariant CNN.
The rule of thumbs is that you can do any changes, but files which are not
marked (by red color in \ref{subsubsec:overview}, e.g.,
``main.py'') will be overwritten by TA in the test stage.

\hfill

We provide a brief command summary for GPU environment setup on RCAC Scholar cluster.
Pay attention that you should use {\bf Python 3.10, CUDA 12.1, PyTorch 2.5.1} for best support.
\begin{lstlisting}[language=bash, breaklines=true]
module load cuda/12.1.0 conda/2024.09-py312
conda create -n DPLClass python=3.10 ipython ipykernel
conda activate DPLClass
conda install pytorch==2.5.1 torchvision==0.20.1 pytorch-cuda=12.1 -c pytorch -c nvidia
conda install scikit-learn seaborn more-itertools
\end{lstlisting}

Note that every time you login to the scholar cluster, you should run the 1st and 3rd commands again. 

\textbf{A GPU is essential for some of the tasks, thus make sure you follow the instruction to set up the GPU environment on ``scholar.rcac.purdue.edu''.} A tutorial is available at\\
{\scriptsize \url{https://www.cs.purdue.edu/homes/ribeirob/courses/Spring2025/howto/cluster-how-to.html}}.

In particular, you should learn how to navigate the \textbf{Slurm} system to request back-end computing nodes equipped with GPUs on the \texttt{gpu} Slurm queue. Ensure you understand how to use the \texttt{slist, squeue, sinteractive, sbatch, scancel} commands. 

\textbf{This time for your convenience, we provide a Slurm batch submission script called \texttt{scholar.sh} in the skeleton code.} You can use it to submit \emph{any} command lines to a GPU computing nodes. For instance, suppose the original command to run a python training script is \texttt{python main.py data}, then if run the following:
\begin{lstlisting}[language=bash, breaklines=true]
sbatch scholar.sh python main.py data
\end{lstlisting}
Then the same job will be submitted to a GPU backend node. 


\subsection*{HW Overview}
\label{subsubsec:overview}
There are two tasks in this homework. The first task, namely Hamiltonian Monte Carlo, will use the codebase named ``\texttt{hw2\_hmcmlp\_skeleton}''. The second task, namely CNN and G-Invariant CNN, will use the codebase named ``\texttt{hw2\_imageclassifier\_skeleton}''.

\newpage

\section{Q2 (4.0 pts): Hamiltonian Monte Carlo for Multi Layer Perceptron in Image classification} In this part, you are going to implement HMC Sampler for learning Multi Layer Perceptron (MLP) weights.
%

% Space.
%
\hfill

% Download skeleton.
%
\noindent \textbf{Skeleton Package:}
%
A skeleton package named ``\texttt{hw2\_hmcmlp\_skeleton}'' is provided on Brightspace.
%
You should be able to download it and use the folder structure provided.
%
%
\noindent The zip file should have the following folder structure:

% Skeleton figure.
%
\tikzstyle{every node}=[draw=black,thick,anchor=west]
\tikzstyle{selected}=[draw=red,fill=red!30]
\tikzstyle{core}=[draw=blue,fill=blue!30]
\tikzstyle{optional}=[dashed,draw=red,fill=gray!50]
%
\begin{tikzpicture}[%
    grow via three points={
        one child at (0.5,-0.7) and two children at (0.5,-0.7) and (0.5,-1.4)
    },
    edge from parent path={(\tikzparentnode.south) |- (\tikzchildnode.west)}
]
%
    \node {hw\homeworknumber\_hmcmlp\_skeleton}
    child {node {NeuralNetwork.py}}
    child {node {mnist.py}}
    child {node {utils.py}}
    child {node [core] {main.py}}
    child {node {minibatcher.py}}
    child {node [selected] {HamiltonianMonteCarlo.py}}
    child {node [selected] {PerturbedHamiltonianMonteCarlo.py}}
    child {node {scholar.sh}};
    \end{tikzpicture}

% Space.
%
\hfill

% Skeleton description.
%
\begin{itemize}
%
\item
    \textbf{hw\homeworknumber\_hmcmlp\_skeleton}
    %
    the top-level folder that contains all the files required in this homework.
%
\item
    \textbf{ReadMe:}
    %
    Your ReadMe should begin with a couple of \textbf{execution commands},
    e.g., ``python hw\homeworknumber.py data'', used to generate the outputs
    you report.
    %
    TA would replicate your results with the commands provided here.
    %
    More detailed options, usages and designs of your program can be followed.
    %
    You can also list any concerns that you think TA should know while running
    your program.
    %
    Note that put the information that you think it's more important at the
    top.
    %
    Moreover, the file should be written in pure text format that can be
    displayed with Linux ``less'' command.
    %
    
%
\item
    \textbf{utils.py:}
    %
    Utility functionalities used in main execution files.

\item
    \textbf{main.py:}
    %
    The \underline{main executable} to run HMC sampling for MLP parameters. You are asked to not to change the code. Even if you change any part of the code, while doing the evaluation the TA will replace it with the original main.py file.
    %


\item
    \textbf{minibatcher.py:}
    %
    Python Module to implement batchifying in MNIST dataset%
    %
\item
    \textbf{mnist.py:} A mnist data structure to load the MNIST dataset.
%
\item
    \textbf{NeuralNetwork.py:} Multi Layer Perceptron model implemented with PyTorch defined for this homework. You should not change the code.
    %
    %
    \item \textbf{HamiltonianMonteCarlo.py:} \underline{\bf You will need to implement} the necessary functions here for developing a Hamiltonian Monte Carlo Sampler Module
    \item \textbf{PerturbedHamiltonianMonteCarlo.py:} \underline{\bf You will need to implement} the necessary functions here for developing a Hamiltonian Monte Carlo Sampler Module where only the last upper layer of the MLP will be sampled
    %
    \item \textbf{scholar.py:} A utility bash script to help you submit the Python process to a GPU compute node.
    
%
\end{itemize}


% =============================================================================
% *****************************************************************************
% -----------------------------------------------------------------------------
% ## Q2
% -----------------------------------------------------------------------------
% *****************************************************************************
% =============================================================================

% Beginning of HMC question.
%
\newpage
\section*{%
    HMC for MLP parameters
}

% Space.
%
\hfill

% Introduction.
%
\noindent
%
Consider both a 1-hidden layer (``Shallow") and a 2-hidden layered (``Deep") Multi Layer perceptrons (MLP). You are going to implement a posterior sampler using Hamiltonian Monte Carlo (HMC) algorithm presented in the class to classify MNIST images. For simplicity purpose, the problem of image classification has been converted into a binary one, instead of multi-class: every image has been labeled either 0 for even digits (0, 2, 4, 6, 8) or 1 for odd digits (1, 3, 5, 7) 
%Consider a Gaussian Mixture Model (GMM) $X \in \mathbb{R}^2$ with $M$ clusters and parameters $\mW = \{\mu_{i}, \sigma_{i}, w_{i}\}_{i = 1}^{M}$, where $\mu_{i} \in \mathbb{R}^2$.
%That is, the GMM descibes a set of points in the 2D plane.
%You are going implement a posterior sampler using the Hamiltion Monte Carlo (HMC) algorithm presented in class.

HMC recap: In general, given a model (say, our MLP) with parameters  $\mW$ and a training dataset $D$.
A Bayesian sampler of this model obtains $m$ samples $\mW_{t} \sim P(\mW|D)$, where $t \in \{0,\ldots,m-1\}$ is the sample index.
To achieve this via HMC, we need two measurements, the {\em potential} energy $U(\mW)$
and the {\em kinetic} energy $K(\mPhi)$, where $\mPhi \sim \mathcal{N}(0, \mR)$ is the
auxiliary momentum in HMC algorithm randomly sampled from zero-mean Gaussian
distribution with covariance matrix $\mR$.
The choice of $\mR$ is left to you.

Given an arbitrary dataset $\mathcal{D}$, we have $$U(\mW) = -\log
P(\mathcal{D}|\mW) + Z_{U},$$ and $$K(\mPhi) = 0.5 \cdot \mPhi^\mathsf{T}
\mR^{-1} \mPhi + Z_{K},$$ where $-\log P(\mathcal{D}|\mW)$ is negative
log-likelihood (mean) of model parameter on dataset $\mathcal{D}$ and $Z_{U},
Z_{K}$ are arbitrary constants.
Thus, we can regard the total energy as $$H(\mW, \mPhi) = -\log
P(\mathcal{D}|\mW) + 0.5 \cdot \mPhi^\mathsf{T} \mR^{-1} \mPhi.$$

The HMC algorithm can be described as \Cref{alg:hmc}:
\begin{algorithm}
\caption{Single Step Sampling of Hamilton Mento Carlo}
\label{alg:hmc}
\begin{algorithmic}
\Require Previous sample $\mW_{t}$, Size of Leapfrog Step $\delta$, Number of
Leapfrog Steps $L$, Covariance $\mR$
\Ensure New sample $\mW_{t + 1}$
\State $\mPhi_{0} \sim \mathcal{N}(0, \mR)$
\State $\mX_{0} = \mW_{t}$
\For{$l = 0, \cdots, L - 1$}
    \State $\mPhi_{\big( l + \frac{1}{2} \big) \delta} = \mPhi_{l \delta} -
    \frac{\delta}{2} \left. \frac{\partial U(\mW)}{\partial \mW}
    \right|_{\mW = \mX_{l \delta}}$
    \State $\mX_{(l + 1) \delta} = \mX_{l \delta} + \delta \mR^{-1}
    \mPhi_{\big( l + \frac{1}{2} \big) \delta}$
    \State $\mPhi_{(l + 1) \delta} = \mPhi_{\big( l +
    \frac{1}{2} \big) \delta} - \frac{\delta}{2} \left. \frac{\partial U(\mW)}
    {\partial \mW} \right|_{\mW = \mX_{(l + 1) \delta}}$
\EndFor
\State $\alpha = \min\big(1, \exp(-H(\mX_{L \delta}, \mPhi_{L \delta}) +
H(\mX_{0}, \mPhi_{0}))\big)$
\If{$\text{Uniform}(0, 1) \leq \alpha$}
    \State $\mW_{t + 1} = \mX_{L \delta}$
\Else
    \State $\mW_{t + 1} = \mW_{t}$
\EndIf
\end{algorithmic}
\end{algorithm}
% Space.
%
\hfill

% Space.
%
\hfill

% Files to work on.
%
%\newpage
\noindent Action Items:
%
Let $\mathbf{W}$ are the weights of the Multi Layer Perceprton, and $\mathcal{D} = \{\mathbf{x_i}, \mathbf{y_i}\}_{i=1}^n$ are the training data ($\mathbf{x_i}$ being the MNIST image and $\mathbf{y_i}$ is the image label). 
After sampling $K$ samples of MLP weights $\mathbf{W^{(1)}}, \mathbf{W^{(2)}}, \ldots, \mathbf{W^{(k)}}$, the Bayesian average model for classification will be,\\
$p(\mathbf{y}|\mathbf{x}; \{\mathbf{x_i}, \mathbf{y_i}\}_{i=1}^{n}) = \frac{1}{K}\sum_{k=1}^{K}p(\mathbf{y}|\mathbf{x}; \mathbf{W}^{(k)} )$ where $ \mW^{(k)} \sim P(\mW | \mathcal{D})$.
    
    We will implement $\mW^{(k)} \sim P(\mW | \mathcal{D})$ by HMC in \texttt{HamiltonianMonteCarlo.py} and \texttt{PerturbedHamiltonianMonteCarlo.py} 
    according to \Cref{alg:hmc}.
    Go through all related modules. Specifically, you should understand \texttt{main.py}, \texttt{utils.py}, \texttt{NeuralNetworks.py}.
    Run the \texttt{main.py} with default arguments: \texttt{python main.py} to run the programs.
% Group all Reliability figures (Alternating Deep/Shallow, MLE/Bayesian/Perturbed, Test)

\begin{figure}[h!]
    \centering
    \subfigure[Deep MLE Train Loss]{\includegraphics[width=0.45\textwidth]{deep_mle_train_loss.jpg}\label{fig:deep_mle_train_loss}}
    \subfigure[Shallow MLE Train Loss]{\includegraphics[width=0.45\textwidth]{shallow_mle_train_loss.jpg}\label{fig:shallow_mle_train_loss}}
    \caption{Train Loss Performance for Deep and Shallow Networks - MLE vs. Bayesian vs. Perturbed}
    \label{fig:loss_performance}
\end{figure}

% Group all Accuracy figures (Alternating Deep/Shallow, MLE/Bayesian/Perturbed)
\begin{figure}[h!]
    \centering
    \subfigure[Deep MLE Test Accuracy]{\includegraphics[width=0.3\textwidth]{deep_mle_test_accuracy.jpg}\label{fig:deep_mle_test_accuracy}}
    \subfigure[Deep Bayesian Test Accuracy]{\includegraphics[width=0.3\textwidth]{deep_bayessian_test_auc.jpg}\label{fig:deep_bayessian_test_accuracy}}
    \subfigure[Deep Perturbed Bayesian Test Accuracy]{\includegraphics[width=0.3\textwidth]{deep_perturbed_bayessian_test_auc.jpg}\label{fig:deep_perturbed_bayessian_test_accuracy}} \\
    \subfigure[Shallow MLE Test Accuracy]{\includegraphics[width=0.3\textwidth]{shallow_mle_test_accuracy.jpg}\label{fig:shallow_mle_test_accuracy}}
    \subfigure[Shallow Bayesian Test Accuracy]{\includegraphics[width=0.3\textwidth]{shallow_bayessian_test_auc.jpg}\label{fig:shallow_bayessian_test_accuracy}}
    \subfigure[Shallow Perturbed Bayesian Test Accuracy]{\includegraphics[width=0.3\textwidth]{shallow_perturbed_bayessian_test_auc.jpg}\label{fig:shallow_perturbed_bayessian_test_accuracy}} 
    \caption{Test Accuracy Results for Deep and Shallow Networks - MLE vs. Bayesian vs. Perturbed}
    \label{fig:accuracy_performance}
\end{figure}

\begin{enumerate}
%
   
\item
    (1.5 pts) Fill in the missing parts of \texttt{get\_sampled\_velocities()}, \texttt{leapfrog()}, \texttt{accept\_or\_reject()} functions in \texttt{HamiltonianMonteCarlo.py}. Go through the comments in the starter code for each function to understand what they are expected to do. 
    
    In short, \texttt{get\_sampled\_velocities()} sample the initial values of velocities $\mPhi_{0}$; \texttt{leapfrog()} implements the update of $\mPhi, \mX$ through leapfrog steps; \texttt{accept\_or\_reject()} implements the acceptance or rejection procedeure in the algorithm based on the kinetic and potential energies; and \texttt{sample()} combine all these three functions in a way to generate $K$ samples of MLP weight parameters by generating initial velocities, calling leapfrog function multiple times to generate new velocities, decide whether to accept or reject new sample and then prepare the samples.    
    
\item (1 pts) Fill in the missing parts of \texttt{get\_sampled\_velocities()}, \texttt{leapfrog()}, \texttt{accept\_or\_reject()} functions in \texttt{PerturbedHamiltonianMonteCarlo.py} in such a way that it only updates the last layers weights and biases through sampling. In previous implementation all the layers had their weights and biases updated.

\item (Introduction to model calibration) In measuring model performance, we do not only care about accuracy or loss, but also care about if the model is "calibrated", which means we want it to output the ground-truth probability.

Consider our MLP model $f_{\mW}$, parametrized by weight parameters $\mW$. $\mathcal{D} = \{\mathbf{x_i}, \mathbf{y_i}\}_{i=1}^n$ are the training data ($\mathbf{x_i}$ being the MNIST image and $\mathbf{y_i}$ is the image label). For a given input $\vx_i$, the prediction $\hat{y}_i$ can be denoted as,
\begin{displaymath}
\hat{y}_i := \argmax_{k \in {0,1}}[f_{\mW}(\vx_i)]_k,
\end{displaymath}
and the prediction probability (confidence) $\hat{p}_i$ can be denoted as,
\begin{displaymath}
\hat{p}_i := \max_{k \in {0,1}}[f_{\mW}(\vx_i)]_k.
\end{displaymath}
The perfect calibrated model is defined as $P(\hat{Y}=Y|\hat{p} = \alpha) = \alpha$. One notion of miscalibration is the difference in expecctation between confidence and accuracy, i..e, $\mathbb{E}_{\hat{p}}(|P(\hat{Y}=Y|\hat{p} = \alpha) - \alpha|)$. To approximate it by finite samples, we divide the probability/confidence interval $[0,1]$ into $M$ equal sized bins $B_1, B_2, \ldots B_M$. For each of the example $\vx_i$ we group them into these bins according to their $\hat{p_i}$ value, i.e., $B_j = \{i: \hat{p}_i\in [\frac{j-1}{M}, \frac{j}{M})\}$. 
Then, for each bucket $B_j$, we find out 
%
\begin{displaymath}
\rho_j := \frac{1}{\lvert B_j\rvert}\sum_{i \in B_j}{\hat{p}_i},
\end{displaymath}
and
\begin{displaymath} \phi_j := \frac{1}{\lvert B_j\rvert}\sum_{i \in B_j}\mathbf{1}[{\hat{y_i} \text{ is the true label}}].
\end{displaymath}
%
We plot these $(\rho_j, \phi_j)$ in the reliability diagram where $X$-axis is for $\rho$ and $Y$-axis is for $\phi$. $\rho_j, \phi_j$ are respectively called the average confidence and average accuracy for bucket $B_j$. We also define Expected Calibration Error (ECE),
%
\begin{displaymath}
ECE := \frac{1}{n}\sum_{j=1}^{M}{\lvert B_j\rvert}{\lvert \rho_j - \phi_j\rvert}.
\end{displaymath}

\item (1.5 pts) To perform this task, we need to understand how \texttt{main.py} works. This script will at first learn the Multi Layer perceptron network using traditional MLE based learning. The default configuration for the MLP is shallow (only 1 hidden layer). The accuracy and losses are plotted in the process. Then after being pre-trained for a fixed number of epochs, more networks will be sampled through HMC sampling and the averaged output from the ensemble will be used in prediction. After that, perturbed version of HMC sampling is done where only the last layers' weights and biases are sampled. These sampled networks are averaged again for another set of predictions. 

Go through the code to understand how it works. After all the predictions using the MLE, HMC sampling and perturbed HMC sampling are done, ROC curves and reliability curves for all the models on their training and test data are plotted for analysis. Your tasks are:

\begin{enumerate}
    \item (0.5pt) Run the \texttt{main.py} for both shallow and deep networks using the \texttt{--depth} argument while running the code. If you run \texttt{python main.py --depth shallow}, the whole procedure will be run for the shallow network, generating loss, accuracy, ROC, and reliability plots. If you run \texttt{python main.py --depth deep}, the whole procedure will be run for the deep network. You need to understand other command line arguments to tune the necessary hyperparameters (learning rate, leapfrog step numbers, leapfrog step size, delta, etc.). You can set different values for these hyperparameters using the command line arguments. Also, to reduce the runtime, using the \texttt{--loaded} argument, you can decide whether you will learn a neural network and save it, or you will load from an already saved network for the procedures.
    
    In the report, include all the plots that will be generated. Also, mention the training and test accuracies for MLE, sampled, and perturb-sampled models. For each of these three models, write their Expected Calibration Error (ECE) and Expected calibration error at 50\% threshold.

    \begin{table}[h!]
    \centering
    \begin{tabular}{|c|p{6cm}|p{6cm}|}
        \hline
        \multirow{2}{*}{Model} & \textbf{Shallow Network} & \textbf{Deep Network} \\
        \cline{2-3}
        & Accuracy / ECE / Threshold & Accuracy / ECE / Threshold \\
        \hline
        \textbf{MLE Model} & 
        Training Accuracy: 0.9815 \par
        Test Accuracy: 0.9644 \par
        ECE: 0.083466 \par
        ECE at 50\%: 0.083466 & 
        Training Accuracy: 0.99688 \par
        Test Accuracy: 0.9786 \par
        ECE: 0.007236 \par
        ECE at 50\%: 0.007236 \\
        \hline
        \textbf{Sampled Model} & 
        Training Accuracy: 0.9108 \par
        Test Accuracy: 0.9034 \par
        ECE: 0.075352 \par
        ECE at 50\%: 0.075352 & 
        Training Accuracy: 0.89177 \par
        Test Accuracy: 0.8792 \par
        ECE: 0.031918 \par
        ECE at 50\%: 0.031918 \\
        \hline
        \textbf{Perturbed Sampled Model} & 
        Training Accuracy: 0.8938 \par
        Test Accuracy: 0.8886 \par
        ECE: 0.078595 \par
        ECE at 50\%: 0.078595 & 
        Training Accuracy: 0.96865 \par
        Test Accuracy: 0.9584 \par
        ECE: 0.002248 \par
        ECE at 50\%: 0.002248 \\
        \hline
    \end{tabular}
    \caption{Training and Test Accuracies, ECE, and ECE at 50\% Threshold for Shallow and Deep Networks}
    \label{tab:accuracy_ece}
\end{table}

The MLE model performs the best in terms of accuracy, but all models show
relatively high ECE, especially in comparison to the deep network models.  

The deep networks show better performance (higher accuracies and lower ECE),
especially with the MLE and Perturbed Sampled models.

While the Sampled and Perturbed Sampled models have lower accuracy compared to
the MLE models, they show lower ECE values, indicating better calibration and
more reliable probability estimates, especially for deep networks.  

In general, the deep networks tend to perform better both in terms of accuracy
and calibration (lower ECE). However, the sampled and perturbed models offer
advantages in terms of calibration, which may make them more useful in certain
applications where well-calibrated probabilities are important.


    \item (0.5 pt) Explain the findings you got from the generated ROC curves and AUC scores. Does introducing Bayesian sampling improve the performances? If we only sample the last layers, but with more steps, how does the performance change?

\begin{figure}[h!]
    \centering
    \subfigure[Deep MLE Test AUC]{\includegraphics[width=0.3\textwidth]{deep_mle_test_auc.jpg}\label{fig:deep_mle_test_auc}}
    \subfigure[Deep Bayesian Test AUC]{\includegraphics[width=0.3\textwidth]{deep_bayessian_test_auc.jpg}\label{fig:deep_bayesian_test_auc}}
    \subfigure[Deep Perturbed Bayesian Test AUC]{\includegraphics[width=0.3\textwidth]{deep_perturbed_bayessian_test_auc.jpg}\label{fig:deep_perturbed_bayessian_test_auc}} \\
    \subfigure[Shallow MLE Test AUC]{\includegraphics[width=0.3\textwidth]{shallow_mle_test_auc.jpg}\label{fig:shallow_mle_test_auc}}
    \subfigure[Shallow Bayesian Test AUC]{\includegraphics[width=0.3\textwidth]{shallow_bayessian_test_auc.jpg}\label{fig:shallow_bayesian_test_auc}}
    \subfigure[Shallow Perturbed Bayesian Test AUC]{\includegraphics[width=0.3\textwidth]{shallow_perturbed_bayessian_test_auc.jpg}\label{fig:shallow_perturbed_bayessian_test_auc}} 
    \caption{Test AUC Performance for Deep and Shallow Networks - MLE vs. Bayesian vs. Perturbed}
    \label{fig:test_auc_performance}
\end{figure}


    \textbf{ROC Curve and AUC Scores:} Comparing to the shallow model, the deep
    model generally have a better performance in terms of AUC scores, as shown
    in \Cref{tab:accuracy_ece}. This can be a result of more neurons
    participating in the classification task, leading to better generalizability
    and computational power. \\

    The ROC curves and AUC scores are shown in \Cref{fig:test_auc_performance}.\\

    The MLE model has the highest AUC score for both deep and shallow models,
    indicating that it has the best performance in terms of classification.

    \textbf{MLE Model:} The MLE model has an AUC score of 0.9976 and 0.9626 for the deep and shallow model respectively. \\
    \textbf{Sampled Model:} The sampled model (sampling on all layers) has an AUC score of 0.9932 and 0.9421 for the deep and shallow model respectively. \\ 
    \textbf{Perturbed Sampled Model:} The perturbed sampled model (sampling on the last layer) has an AUC score of 0.9971 and 0.9582 for the deep and shallow model respectively. \\
    
    \textbf{Does Bayesian sampling improve performance?} \\

    Not really. Because the AUC score of the sampled model is lower than the MLE
    model.  This can be due to the fact that the sampled model is not focused on
    learning the data, but rather on the overall distribution, which resulted in
    a loss of AUC score. \\.

    \textbf{Does sampling only the last layers with more steps improve performance?} \\

    Yes. The perturbed sampled model has a higher AUC score than the sampled
    model for both deep and shallow models. This can be due to the fact that the
    perturbed sampled model is more focused on the last layers, which serves as
    a regularization mechanism to prevent overfitting, leading to a better AUC
    score.  \\
    
    \item (0.5 pt) Explain the findings you got from the generated Reliability curves and ECE scores. Does introducing Bayesian sampling improve the performances? If we only sample the last layers, but with more steps, how does the performance change?

\begin{figure}[h!]
    \centering
    \subfigure[Deep MLE Test Reliability]{\includegraphics[width=0.3\textwidth]{deep_mle_test_reliablity.jpg}\label{fig:deep_mle_test_reliablity}}
    \subfigure[Deep Bayesian Test Reliability]{\includegraphics[width=0.3\textwidth]{deep_bayessian_test_reliablity.jpg}\label{fig:deep_bayesian_test_reliablity}}
    \subfigure[Deep Perturbed Bayesian Test Reliability]{\includegraphics[width=0.3\textwidth]{deep_perturbed_bayessian_test_reliablity.jpg}\label{fig:deep_perturbed_bayessian_test_reliablity}} \\
    \subfigure[Shallow MLE Test Reliability]{\includegraphics[width=0.3\textwidth]{shallow_mle_test_reliablity.jpg}\label{fig:shallow_mle_test_reliablity}}
    \subfigure[Shallow Bayesian Test Reliability]{\includegraphics[width=0.3\textwidth]{shallow_bayessian_test_reliablity.jpg}\label{fig:shallow_bayesian_test_reliablity}}
    \subfigure[Shallow Perturbed Bayesian Test Reliability]{\includegraphics[width=0.3\textwidth]{shallow_perturbed_bayessian_test_reliablity.jpg}\label{fig:shallow_perturbed_bayessian_test_reliablity}} 
    \caption{Test Reliability Performance for Deep and Shallow Networks - MLE vs. Bayesian vs. Perturbed}
    \label{fig:test_reliability_performance}
\end{figure}

    % Fill in the blanks with your findings
    \textbf{Reliability Curve and ECE Scores:} The reliability curve and ECE
    scores are shown in \Cref{fig:test_reliability_performance}. The ECE scores
    are shown in \Cref{tab:accuracy_ece}. \\

    \textbf{MLE Model:} The MLE model has an ECE score of 0.083466 and 0.007236 for the shallow and deep model respectively. \\
    \textbf{Sampled Model:} The sampled model (sampling on all layers) has an ECE score of 0.075352 and 0.031918 for the shallow and deep model respectively. \\
    \textbf{Perturbed Sampled Model:} The perturbed sampled model (sampling on the last layer) has an ECE score of 0.078595 and 0.002248 for the shallow and deep model respectively. \\
    
    \textbf{Does Bayesian sampling improve performance?} \\

    Yes. More specifically, the deep perturbed sampled model has a much lower
    ECE score than the sampled model and the MLE model. This is also illustrated
    in \Cref{fig:deep_perturbed_bayessian_test_reliablity}, where confidence and
    accuracy are well-calibrated, and forms an almost perfect line. \\

    \textbf{Does sampling only the last layers with more steps improve performance?} \\

    Yes. Sampling on all layers worsens calibration, as seen in the higher ECE
    score (0.031918). Perturbing the last layer drastically improves
    calibration, leading to a very low ECE of 0.002248, which is the best
    calibration performance among all models.

\end{enumerate}

\end{enumerate}

% Table example.
%


\newpage
\section{Q3 (4.0 pts): Image classifiers with CNN and G-Invariant CNN} 
In this task, you will implement a CNN, a G-invariant CNN and various tasks to better understand its inner workings.
%

\subsubsection*{Q3.a HW Overview}
\label{subsubsec:overview}

% Download skeleton.
%
\noindent \textbf{Skeleton Package:}
%
A skeleton package named ``\texttt{hw2\_imageclassifier\_skeleton}'' is provided on Brightspace.
%
You should be able to download it and use the folder structure provided.
%
%
\noindent The zip file should have the following folder structure:

 \hfill

\tikzstyle{every node}=[draw=black,thick,anchor=west]
\tikzstyle{selected}=[draw=red,fill=red!30]
\tikzstyle{core}=[draw=blue,fill=blue!30]
\tikzstyle{optional}=[dashed,draw=red,fill=gray!50]
\begin{tikzpicture}[%
  grow via three points={one child at (0.5,-0.7) and
  two children at (0.5,-0.7) and (0.5,-1.4)},
  edge from parent path={(\tikzparentnode.south) |- (\tikzchildnode.west)}]
  \node {hw\homeworknumber\_imageclassifier\_skeleton}
    % child { node [selected] {homework/template}
    %   child { node [selected] {cnn.py}}
    %   child { node [selected] {mlp.py}}
    %   child { node [selected] {optimizers.py}}
    %   child { node [selected] {subspace.py}}
    %   child { node [optional] {any\_others.py}}
    % }
    % child [missing] {}
    % child [missing] {}
    % child [missing] {}
    % child [missing] {}
    % child [missing] {}
    child { node [selected] {ReadMe}}
    child { node {download.py}}
    % child { node {datasets.py}}
    %child { node {structures.py}}
    child { node [selected] {models.py}}
    child { node {optimizers.py}}
    child { node [core] {main.py}}
    child { node {interface.sh}}
    child { node {visualize.py}}
    child { node {scholar.sh}};
\end{tikzpicture}

\hfill

\begin{itemize}
% \item \textbf{[your\_purdue\_login]\_hw\homeworknumber}: the top-level folder that contains all the files
%           required in this homework. You should replace the file name with your
%           name and follow the naming convention.
\item \textbf{hw\homeworknumber\_imageclassifier\_skeleton}: the top-level folder that contains all the files
          required in this task.


\item \textbf{ReadMe}: Your ReadMe should begin with a couple of \textbf{example commands}, e.g., "python hw\homeworknumber.py data", used to generate the outputs you report. TA would replicate your results with the commands
          provided here. More detailed options, usages and designs of your
          program can be followed. You can also list any concerns that you
          think TA should know while running your program. Note that put the
          information that you think it's more important at the top. Moreover,
          the file should be written in pure text format that can be displayed
          with Linux "less" command.
          You can refer to interface.sh for an example.

\item \textbf{main.py}: The \underline{main executable} to run training with CNN and G-invariant CNN.
\item \textbf{download.py}: The \underline{executable script} to download all essential data for this homework.
\item \textbf{visualize.py}: The \underline{executable script} to render plots for your results.
          It can give you a better understanding of your implementations.

\item \textbf{interface.sh}: The executable bash script to give you examples of main.py usage. It also works as an example for writing ReadMe.

%\item \textbf{structures.py}: The module defines different dataset transformations for this homework.
%          It defines two groups of image operations: rotation and flip.
%          Besides, it is also an \underline{executable script} to generate invariant basis for this homework.

\item \textbf{models.py}: \underline{\bf You will need to implement} the necessary functions in this file as homework. This module defines the CNN and G-Invariant CNN models. 

\item \textbf{optimizers.py}: The module defines the customized optimizers for this homework.
        An existing SGD implementation is already provided for you.

\item \textbf{scholar.py:} A utility bash script to help you submit the Python process to a GPU compute node.


The module that you are going to develop:
\begin{itemize}
\item \textbf{models.py}
\end{itemize}
The detail will be provided in the task descriptions. All other modules are just there for your convenience. It is not requried to use them, but exploring that will be a good practice of re-using code. Again, you are welcome to architect the package in your own favorite. For instance, adding another module, called \texttt{utils.py}, to facilitate your implementation.

\end{itemize}


\subsubsection*{Q3.b Data: MNIST}

You are going to conduct a simple classification task, called MNIST (\url{https://huggingface.co/datasets/ylecun/mnist}). It classifies images of hand-written digits (0-9). Each example thus is a \(28 \times 28\) image. 
\begin{itemize}
\item The full dataset contains 60k training examples and 10k testing examples.
\item We provide \textbf{download.py} that will automatically download the data. Make sure that torchvision library is available.
\end{itemize}


\noindent Related Modules: 
\begin{itemize}
% \item hw\homeworknumber\_minibatch.py 
% \item my\_neural\_networks/optimizers.py
% \item (to modify) template/model.py
% \item (to modify) template/subspace.py
% \item (to create) my\_neural\_networks/CNN\_networks.py
% \item (to create)  my\_neural\_networks/shuffled\_labels.py
\item main.py 
\item models.py
\end{itemize}


\newpage
\subsubsection*{Q3.c Action Items:}
\begin{enumerate}


\item {\bf (0.5 pt)} (In the code) Follow the  the Pytorch implementation CNNNet given in lecture\\
{\small \url{https://www.cs.purdue.edu/homes/ribeirob/courses/Spring2025/lectures/07cnn/CNNs.html}}\\
to create a CNN with 2 convolutional layers with max pooling in {\bf models.py} using default parameters (convolution kernel size 5, stride 1; padding size 3; pooling kernel size 2, stride 2) followed by 3 fully-connected linear layers (your code should be able to automatically calculate for the input dimension of the first linear layer when kernel size and stride is changing).\\
{\bf Optimize using the standard SGD}. 
Run {\bf main.py} with command line argument \verb|--cnn| so it will run the CNN code. (\verb|python main.py --cnn|, one example to run with GPU is \verb|python main.py --cnn|\\ \verb|--batch-size 100 --device cuda|, see full example in \texttt{interface.sh}) \\
{\bf (In the PDF report on Gradescope)} 
Describe the neural network architecture and its hyper-parameters: layers, type of layer, number of neurons on each layer, the activation functions used, and how the layers connect. 

The convolutional neural network (CNN) consists of two convolutional layers 
followed by fully connected layers. The detailed architecture and hyperparameters are 
described as follows:

\begin{itemize}
    \item \textbf{Input:} The input to the network is a grayscale image of size 
    \(28 \times 28\).
    
    \item \textbf{Convolutional Layers:} 
    \begin{itemize}
        \item \textbf{First Convolutional Layer:} Applies \(C_1\) filters of 
        size \(5 \times 5\) with a stride of 1 and padding to preserve spatial 
        dimensions. A ReLU activation function is used.
        
        \item \textbf{Second Convolutional Layer:} Applies \(C_2\) filters of 
        size \(5 \times 5\) followed by ReLU activation.
        
        \item \textbf{Pooling:} A max-pooling layer of size \(2 \times 2\) is 
        applied after each convolutional layer to reduce spatial dimensions.
    \end{itemize}
    
    \item \textbf{Flattening:} The output feature maps are flattened into a 
    vector before passing to the fully connected layers.

    \item \textbf{Fully Connected Layers:} 
    \begin{itemize}
        \item First dense layer: 300 neurons, ReLU activation.
        \item Second dense layer: 100 neurons, ReLU activation.
        \item Output layer: 10 neurons, softmax activation for classification.
    \end{itemize}
    
    \item \textbf{Layer Connectivity:} The overall flow of data through the 
    network follows:
    
    \begin{align*}
    x &\rightarrow \text{Conv}_1 \rightarrow \text{ReLU} \rightarrow 
    \text{Pool}_1 \rightarrow \text{Conv}_2 \rightarrow \text{ReLU} \\
    &\rightarrow \text{Pool}_2 \rightarrow \text{Flatten} \rightarrow 
    \text{FC}_1 \rightarrow \text{ReLU} \rightarrow \text{FC}_2 \rightarrow 
    \text{ReLU} \rightarrow \text{Output}
    \end{align*}
    
    \item \textbf{Optimization and Hyperparameters:} 
    \begin{itemize}
        \item \textbf{Optimizer:} Stochastic Gradient Descent (SGD).
        \item \textbf{Batch size:} 100.
        \item \textbf{Regularization:} Not applied in this implementation.
        \item \textbf{Dropout:} Not included in this implementation.
    \end{itemize}
\end{itemize}

\item {\bf (0.5 pt)} For a  $k \times k$  filter, the CNN considers $k \times k$
image patches.  These image patches overlap according to stride, which is by how
much each block must be separated horizontally and vertically.  If $k$ is not a
multiple of the image height of width, we will need padding (increase image
height (width) by adding a row (column) of zeros).
% Modify these filters to (a) $3 \times 3$ with stride 3, and (b) $14 \times 14$
% with stride 1.
Modify the command line arguments \verb|--kernel| and \verb|--stride| to (a) $3
\times 3$ with stride 3, and (b) $14 \times 14$ with stride 1.  
(\verb|python main.py --kernel kernel --stride|\\ \verb|stride --cnn --batch-size 100|, 
see full example in \texttt{interface.sh}) In the {\bf PDF report on Gradescope},
show the test accuracy of the classifier over training and test data for items
(a) and (b)..  Discuss your findings. Specifically, what could be the issue of
having (a) $3\times 3$ filters with stride 3 and (b) $14 \times 14$ filters? \\

The following table presents the test accuracy of the classifier when using 
(a) $3\times 3$ filters with stride 3 and (b) $14 \times 14$ filters.

\begin{table}[h]
    \centering
    \renewcommand{\arraystretch}{1.2}
    \setlength{\tabcolsep}{4pt}
    \begin{tabular}{|c|c|c|c|p{4.5cm}|}
        \hline
        \textbf{Filter Size \& Stride} & \textbf{Train Loss} & \textbf{Train Acc (\%)} & \textbf{Test Acc (\%)} & \textbf{Potential Issues} \\
        \hline
        $3\times 3$, Stride 3 & 1.5439 & 46.13 & 46.49 & Large stride lowers resolution, discarding key details, which may hurt performance. \\
        \hline
        $14\times 14$ & 0.3225 & 87.82 & 87.82 & Filters cover too much area, limiting local feature extraction and leading to less informative representations. \\
        \hline
    \end{tabular}
    \caption{Test Accuracy and Issues with Different Filter Sizes}
    \label{tab:filter_analysis}
\end{table}

From the results, we observe that:
\begin{itemize}
    \item \textbf{$3\times 3$ filters with stride 3:} The large stride 
    reduces spatial resolution significantly, possibly discarding 
    useful fine-grained details and impacting classification performance.
    
    \item \textbf{$14\times 14$ filters:} Filters are too large, covering 
    nearly the entire image. This limits the ability to capture meaningful 
    local structures, leading to overly generalized feature representations. 
    Despite high accuracy, it may rely on global patterns rather than useful 
    discriminative features. Another concern is prolonged training time.
\end{itemize}

This analysis highlights the importance of selecting appropriate filter sizes 
and strides for effective feature extraction while preserving essential image details.


\item {\bf (1.5 pt, equally distributed)} Deep neural networks generalization performance is not related to the network's inability to overfit the data. Rather, it is related to the solutions found by SGD. 
In this part of the homework we will be testing that hypothesis.
Please use {\bf 100 epochs} in the following questions.
\begin{enumerate}
\item Using the provided code with (kernel size $5$ and stride $1$), show a plot {\bf (in the PDF report on Gradescope)} with two curves: the training accuracy  and testing accuracy, with the x-axis as the number of epochs.  \\(\verb|python visualize.py --cnn|)

\begin{figure}[h!]
    \centering
    \begin{minipage}[b]{\textwidth}
        \centering
        \includegraphics[width=\textwidth]{cnn_acc.png}
        \caption{CNN Accuracy}
        \label{fig:cnn_acc}
    \end{minipage}
    \hspace{0.05\textwidth}
    \begin{minipage}[b]{0.4\textwidth}
        \centering
        \includegraphics[width=\textwidth]{shuffle_cnn_acc.png}
        \caption{Shuffle CNN Accuracy}
        \label{fig:shuffle_cnn_acc}
    \end{minipage}
\end{figure}

The plot is shown in the left-most figure of \Cref{fig:cnn_acc}. The training
accuracy increases steadily over the 100 epochs, reaching a high value of
99.5\%.

\item Now consider an alternative task: randomly shuffle the target labels (kernel size $5$ and stride $1$), so that the image (handwritten digit) and its label are unrelated. Show a plot with two curves: the training accuracy and validation accuracy, with the x-axis as the number of epochs.
Add command line parameter \verb|--shuffle-label| so it will ran this shuffle-label CNN experiment with parameter . We use learning rate $1e-2$ in this case.
(\verb|python main.py --shuffle-label --cnn --lr 1e-2|, and then \verb|python visualize.py --shuffle| see full example in \texttt{interface.sh}).\\
{\bf (In the PDF report on Gradescope)} What can you conclude about the ability of this neural network to overfit the data? Would you say that the inductive bias of a CNN can naturally ``understand images'' and will try to classify all hand-written ``0''s as the same digit?

Based on the figure and the results, we can conclude that the CNN model trained
on the MNIST dataset with shuffled labels fails to generalize effectively. This
is indicated by the low accuracy by the end of 100 epochs (11.3\%). This
suggests that There exists a correct association between images and labels, and
the presence of this association is crucial for the model to learn the true
underlying patterns and generalize effectively. Therefore, we can conclude that
the unshuffled model generalizes well, and learns to extract relevant features
from the input images and generalize well to unseen samples.

\begin{figure}[ht]
    \centering
    \subfigure[Accuracy (No Shuffle)]{
        \includegraphics[width=0.45\textwidth]{acc_noshuffle_100_cnn_5_1.png}
    }
    \hfill
    \subfigure[Accuracy (Shuffle)]{
        \includegraphics[width=0.45\textwidth]{acc_shuffle_100_cnn_5_1.png}
    } \\
    \subfigure[Loss (No Shuffle)]{
        \includegraphics[width=0.45\textwidth]{loss_noshuffle_100_cnn_5_1.png}
    }
    \hfill
    \subfigure[Loss (Shuffle)]{
        \includegraphics[width=0.45\textwidth]{loss_shuffle_100_cnn_5_1.png}
    }
    \caption{Accuracy and Loss Interpolated Against Model Weights: 
    Comparison with and without Shuffling}
    \label{fig:interp_weights_shuffle}
\end{figure}

\item Using the approach to interpolate the initial (random) and final parameters seen in class\\
{\scriptsize\url{https://www.cs.purdue.edu/homes/ribeirob/courses/Spring2025/lectures/05optimization/Neural_Network_Trainingv2.html}}.\\
{\bf (In the PDF report on Gradescope)}
show the ``flatness'' plot of the original task and the task with shuffled labels over the training data. (We do not provide framework, and you should code by yourself from ground based on \verb|main.py|.) Can you conclude anything from these plots about the possible generalization performance of the two models?
You just need to include the plots in the PDF (code is not required).

\Cref{fig:interp_weights_shuffle} shows the interpolated accuracy and loss
curves for the CNN model trained on the MNIST dataset with and without shuffled
labels. The following observations can be made:

When no shuffling is involved, the model generalizes well, as witnessed by the
training and test accuracy curves closely tracking each other. In this case, it
is fair to conclude that the inductive bias of the CNN model is effective in
understanding images and classifying handwritten digits accurately. The model
learns to extract relevant features from the input images and generalize well to
unseen samples.

When shuffling is involved, the model struggles to generalize, as indicated by
the low accuracy achieved at the end of the training (11.3\%). The increasing
gap between training loss and validation loss further suggests that the model
has overfit the training data and fails to generalize to unseen samples. 


\end{enumerate}

\item {\bf (1.5 pts)}
 Implement a CG-CNN (G-invariant CNN) (kernel size 5, stride 1; all settings are the same as CNN) that is jointly invariant to the following transformation groups: Horizontal \& vertical image flips and 90$^\circ$ rotations. Implement the missing functions in \texttt{models.py}.
 You should first run \verb|python structures.py --size|\\
 \verb|size| to achieve essential
 eigenvectors of invariant subspace for those transformation groups where size should be a proper value for your CG-CNN.
 It will save basis in file ``rf-size.npy''.
 Then, in ``models.CGCNN'', you should load basis from the file, and implement
 a CGCNN with 2 G-invariant CNN layers constructed by given eigenvectors with proper pooling and a 3-layer MLP. In this part, you should run CNN and CGCNN on the rotated-flip test data with \verb|python main.py --cnn --rot-flip| and \\\verb|python main.py --cgcnn --rot-flip| (Full example is in \texttt{interface.sh}). Report the training and test accuracy results for CNN and CGCNN in the {\bf PDF report on Gradescope}. Can you see the difference between training and test performance for both models and explain the reason?

\begin{table}[ht]
\centering
\begin{tabular}{|c|c|c|}
\hline
\textbf{Model} & \textbf{Train Accuracy} & \textbf{Test Accuracy} \\
\hline
CNN & 0.976950 & 0.344800 \\
\hline
CGCNN & 0.729417 & 0.498700 \\
\hline
\end{tabular}
\caption{Training and Test Accuracy Results for CNN and CGCNN}
\label{table:accuracy_results}
\end{table}

\paragraph{Analysis}

The CNN achieves a high training accuracy of 97.70\%, but its test accuracy
drops significantly to 34.48\%. This suggests severe \textbf{overfitting}, where
the model memorizes the training data but fails to generalize to unseen samples.
The large gap between training and test accuracy indicates poor generalization,
possibly due to insufficient regularization or an overly complex model.

On the other hand, CGCNN exhibits a more balanced performance, with invariant
kernels over the 16 transformations, and an averaging operation performed on the
kernel values prior to the fully connected layer. The model achieves a training
accuracy of 72.94\% and a test accuracy of 49.87\%. Comparing to the CNN, the
training accuracy is lower by the end of the 50 epochs, possibly due to the
averaging operation in the forward pass causing the model to learn more
generalizable features. On the other hand, the test accuracy of CGCNN is higher
thant that of CNN, suggesting that the CGCNN is less prone to overfitting than
the CNN, and is more invariant to transformations such as rotation and flipping.

However, the CGCNN model still struggles to generalize effectively, likely due
to the following reasons:

\begin{enumerate}
    \item \textbf{Overfitting}: The model could be memorizing the training data
    rather than learning generalizable features, leading to poor performance on
    unseen test data.
    \item \textbf{Lack of Regularization}: The model may not be sufficiently
    regularized, meaning it lacks mechanisms like dropout or weight decay to
    help prevent overfitting.
    \item \textbf{Inherent Symmetry in Data}: The MNIST dataset contains numbers
    that are inherently invariant to certain transformations (e.g., rotations
    and flips, such as number 6 and 9). The CGCNN may not be effectively
    leveraging these symmetries to improve generalization.
\end{enumerate}

\end{enumerate}



% =============================================================================
% *****************************************************************************
% -----------------------------------------------------------------------------
% ## Q4
% -----------------------------------------------------------------------------
% *****************************************************************************
% =============================================================================



% =============================================================================
% *****************************************************************************
% -----------------------------------------------------------------------------
% ## Submission Instruction
% -----------------------------------------------------------------------------
% *****************************************************************************
% =============================================================================

%\clearpage

% Beginning of submission.
%
\subsection*{Submission Instructions}
%
Please read the instructions carefully.
%
Failed to follow any part might incur some score deductions.

% Space.
%
\hfill

% PDF.
%
\subsection*{\bf PDF upload}
%
The report PDF must be uploaded on Gradescope.

% Space.
%
\hfill

% Code.
%
\subsection*{\bf Code upload}
%
\noindent \textbf{Naming convention}:
%
[firstname]\_[lastname]\_hw\homeworknumber
\\
%
All your submitting code files, a ReadMe, should be included in one folder. Create a folder [firstname]\_[lastname]\_hw\homeworknumber \ and put both the implemented skeleton folders inside it. Then tar it.\\
%
The folder should be named with the above naming convention.
%
For example, if my first name is ``Bruno'' and my last name is ``Ribeiro'',
then for Homework \homeworknumber, my file name should be
``bruno\_ribeiro\_hw\homeworknumber''.

% Space.
%
\hfill

% Pack your submission.
%
\noindent \textbf{Tar your folder:}
%
[firstname]\_[lastname]\_hw\homeworknumber.tar.gz
\\
%
Remove any unnecessary files in your folder, such as training datasets.
%
Make sure your folder structured as the tree shown in Overview section.
%
Compress your folder with the the command:
%
\textbf{
    tar -czvf bruno\_ribeiro\_hw\homeworknumber.tar.gz bruno\_ribeiro\_hw\homeworknumber
.}

% Space.
%
\hfill

% Submit.
%
\noindent \textbf{Submit:}
%
{\bf Brightspace}
\\
%
Please submit your compressed file on \textbf{Brightspace}.
%
Please make sure you didn't use any library/source explicitly forbidden to use.
%
If such library/source code is used, you will get 0 pt for the coding part of
the assignment.
%
If your code doesn't run on scholar.rcac.purdue.edu, then even if it compiles
in another computer, your code will still be considered not-running and the
respective part of the assignment will receive 0 pt.


% =============================================================================
% *****************************************************************************
% -----------------------------------------------------------------------------
% ## End Of Document
% -----------------------------------------------------------------------------
% *****************************************************************************
% =============================================================================


% End of document.
%
\end{document}
