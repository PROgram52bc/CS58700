\documentclass[25pt, a0paper, portrait]{tikzposter}
\usepackage[
    backend=biber,   % Use biber to compile
    style=numeric,   % Or 'authoryear', etc.
]{biblatex}
\usepackage{amsmath}
\usepackage{amssymb}
\usepackage{booktabs}
\usepackage{graphicx}
\usepackage{subcaption}
\usepackage{float}
\usepackage{pifont}  % for cross
\usepackage{xcolor}  % for color


\definecolorstyle{purdueColorStyle}{
    \definecolor{purduegold}{HTML}{B19447}
    \definecolor{purdueblack}{HTML}{000000}
}{
    \colorlet{backgroundcolor}{white}
    \colorlet{titlebgcolor}{white}
    \colorlet{titlefgcolor}{purdueblack}
    \colorlet{blocktitlebgcolor}{purduegold}
    \colorlet{blocktitlefgcolor}{purdueblack}
    \colorlet{blockbodybgcolor}{white}
    \colorlet{blockbodyfgcolor}{black}
    \colorlet{innerblocktitlebgcolor}{purduegold}
    \colorlet{innerblocktitlefgcolor}{purdueblack}
    \colorlet{innerblockbodybgcolor}{white}
    \colorlet{innerblockbodyfgcolor}{black}
}

\newcommand{\cmark}{\textcolor{green!70!black}{\ding{51}}} % green checkmark
\newcommand{\xmark}{\textcolor{red!70!black}{\ding{55}}}   % red cross

\addbibresource{poster.bib}
% \bibliographystyle{plainnat}

\title{\parbox{\linewidth}{%
            \centering
            \textbf{Compositional Neural Reasoning} \\
            \textit{via Pretrained Perception and Program Control}
}}
\author{Haotian Deng \hspace{1em} Purdue ID: deng254}
\institute{Purdue University}
\titlegraphic{\mbox{\includegraphics[width=0.7\linewidth]{CompSci_V-Full-CMYK.jpg}}}


\settitle{
\centering
\vbox{
\hbox{
    % Left side (Title, Author, Institute)
    \begin{minipage}[h]{0.65\linewidth}
        \color{titlefgcolor}
        {
        \centering
        \bfseries \Huge \sc \@title \par}
        \vspace{1em}
        
        {
        \centering
        \huge \@author \par}
        \vspace{1em}
        
        {
        \centering
        \LARGE \@institute \par}
    \end{minipage}%
    \hfill
    % Right side (Logo)
    \begin{minipage}[h]{0.3\linewidth}
        \raggedleft % Align right
        \@titlegraphic
    \end{minipage}
    \vspace{-5em} % Adjust vertical space
}
}
}

% \definetitlestyle{ShortPurdueTitle}{
%     width=\textwidth,
% }{
%     \node[anchor=north, inner sep=0.8cm, line width=2pt, fill=titlebgcolor, rounded corners] 
%     at (\titleposleft+\titleposright/2, \titlepostop) {%
%         \begin{minipage}[c]{0.75\titlewidth}
%             \centering
%             {\Huge \textbf{Compositional Neural Reasoning}} \\[-0.4em]
%             {\large \textit{via Pretrained Perception and Program Control}} \\[0.4em]
%             {\normalsize \textbf{David Deng}} \\[-0.4em]
%             {\normalsize Purdue ID: deng254} \\[0.4em]
%             {\Large \TP@maketitle@name} \\[-0.4em]
%             {\normalsize \TP@maketitle@institute}
%         \end{minipage}%
%         \hfill
%         \begin{minipage}[c]{0.20\titlewidth}
%             \centering
%             \includegraphics[height=2cm]{CompSci_H-Full-CMYK.jpg}
%         \end{minipage}
%     };
% }

% \usetheme{Simple}
% \usetheme{Rays}
% \usetheme{Board}
\usetheme{Default}
\usecolorstyle{purdueColorStyle}
% \usecolorstyle{Britain}
% \usecolorstyle{Sweden}
% \usecolorstyle{Australia}
% \usetitlestyle{Envelope}
\usetitlestyle{Filled}
% \usetitlestyle{VerticalShading}

\begin{document}
\maketitle

% Introduction
\block{Introduction}{
    
   This project investigates the role of partially initializing weights with
   pretrained perception modules in enabling compositional reasoning in neural
   networks. We learn a function \textbf{$p(\text{y} \mid \text{x}, \text{tag})$}, where
   \textbf{x} is a $2\times 2$ grid of digits and $\textbf{tag}$ specifies the
   operation (e.g., row sum). By separating perception (digit recognition) from
   symbolic control (program heads), we enable compositional reasoning in grid
   tasks.

}

\begin{columns}
\column{0.5}

\block{Motivation}{
   Many complex problems can be naturally \textbf{decomposed} into compositions
   of \textbf{simpler subprograms}, often with known or interpretable
   structures. By separately pretraining and initializing well-understood
   components, such as perceptual modules for digit recognition, we investigate
   how such decomposition affects training stability, convergence speed, and
   test accuracy. This approach leverages \textbf{existing knowledge} through
   pretrained modules, which can be \textbf{beneficial} for \textbf{complex
   tasks} requiring both perception and reasoning.
}

\block{Problem Statement}{
    \textbf{Task:} Learn a function that, given a grid of digits and a tag specifying a selection rule, computes the sum of the selected digits. \\
    \textbf{Input:} A digit grid (e.g., $2 \times 2$) and a tag indicating which subset of digits to sum.  \\
    \textbf{Output:} A single scalar: the sum of the selected digits.
}
% Dataset
\block{Dataset}{
    \textbf{Input 1: MNIST Digits} arranged in $2 \times 2$ grids. \\
    \textbf{Input 2: Tag} indicates which cells are to be summed (See Figure 2). \\
    \begin{minipage}[h]{0.45\linewidth}
        \centering
        \includegraphics[scale=0.8,width=\linewidth]{problem_statement.png}

        Figure 1: Illustration of the problem setup, showing the input $2 \times
        2$ grid of digits and the corresponding output by summing along each
        row/column.

    \end{minipage}
    \hspace{0.02\linewidth} % small horizontal space between images
    \begin{minipage}[h]{0.45\linewidth}
        \centering
        \scalebox{2}{
            \usetikzlibrary{positioning}
\begin{tikzpicture}[every node/.style={font=\small}]
    % Outer 2x2 grid
    \foreach \i in {0,1} {
        \foreach \j in {0,1} {
            \begin{scope}[shift={(\j*4, -\i*4)}]
                % Inner 2x2 grid
                \draw (0,0) rectangle (2,2);
                \draw (1,0) -- (1,2);
                \draw (0,1) -- (2,1);
                % Label digits
                \node (d\i\j) at (1,1) {}; % For relative positioning
                \node at (0.5,1.5) {d0};
                \node at (1.5,1.5) {d1};
                \node at (0.5,0.5) {d2};
                \node at (1.5,0.5) {d3};
                
                % Highlight depending on program id
                \ifnum\i=0
                    \ifnum\j=0
                        % Program 0: top row (d0, d1)
                        \draw[red,very thick,rounded corners] (0,1) rectangle (2,2);
                    \else
                        % Program 1: bottom row (d2, d3)
                        \draw[red,very thick,rounded corners] (0,0) rectangle (2,1);
                    \fi
                \else
                    \ifnum\j=0
                        % Program 2: left column (d0, d2)
                        \draw[red,very thick,rounded corners] (0,0) rectangle (1,2);
                    \else
                        % Program 3: right column (d1, d3)
                        \draw[red,very thick,rounded corners] (1,0) rectangle (2,2);
                    \fi
                \fi
            \end{scope}
        }
    }

    % Program ID labels relative to grids
    \node[below=0.5cm of d10] {Tag = 2};
    \node[below=0.5cm of d00] {Tag = 0};
    \node[below=0.5cm of d11] {Tag = 3};
    \node[below=0.5cm of d01] {Tag = 1};

\end{tikzpicture}
        }
        Figure 2: Tag indicates which cells to sum. The tag is
        passed to the program head, which helps the model select the appropriate
        cells for summation.
    \end{minipage}
}


% Proposed Approach
\block{Proposed Approach}{
    \textbf{Model Overview} The model predicts the \textbf{sum} of selected digits in a $2 \times 2$ grid based on the tag.

    \begin{minipage}[h]{0.45\linewidth}
        \centering
        \scalebox{.85}{
            \begin{tikzpicture}[node distance=1.8cm, every node/.style={align=center}, >=latex']

\node (input) [draw, rounded corners, minimum width=2.5cm, minimum height=1cm]
{\textbf{Input 1: Digits}\\ Grid of Digits};

\node (perception) [draw, rounded corners, below of=input, minimum width=2.5cm,
minimum height=1cm, yshift=-1.5cm] {\textbf{Perception}\\ Digit Recognizer};

\node (tag) [draw, rounded corners, right of=input, xshift=7.5cm, minimum
width=2.5cm, minimum height=1cm] {\textbf{Input 2: Tag}\\ Row/Col Selection};

\node (programhead) [draw, rounded corners, below of=tag, minimum
width=2.5cm, yshift=-1.5cm, minimum height=1cm] {\textbf{Program Head}\\ (Pointer Network)};

\node (symbolic) [draw, rounded corners, below of=perception, yshift=-1.5cm,
minimum width=4cm, xshift=4.25cm, minimum height=1cm] {\textbf{Symbolic Reasoning}\\
(Symmetrized Selection)};

\node (output) [draw, rounded corners, below of=symbolic, yshift=-1.2cm, minimum
width=2.5cm, minimum height=1cm] {\textbf{Output}\\ Sum Prediction};

% Arrows
\draw[->] (input) -- (perception);
\draw[->] (tag) -- (programhead);
\draw[->] (perception) -- (symbolic);
\draw[->] (programhead) -- (symbolic);
\draw[->] (symbolic) -- (output);
\end{tikzpicture}%
        }
        \\
        Figure 3: Architecture Illustration. 
    \end{minipage}%
    \begin{minipage}[h]{0.51\linewidth}
        \centering
        \begin{itemize}
            \item \textbf{Perception:} Processes each $28 \times 28$ cell independently, outputs digit probabilities ($0$--$9$).
            \item \textbf{Program Head:} Predicts soft attention over grid cells.
            \item \textbf{Symbolic Reasoning:} Normalize outer product to ensure \textbf{permutation invariance}.
            \item \textbf{Sum Prediction:} Symmetrize masked attention to combine probabilities over sums ($0$--$18$).
        \end{itemize}
    \end{minipage}%

    % \textbf{Training Objective:} Minimize the negative log-likelihood of the true sum label.

    \vspace{0.5cm}

    \begin{minipage}[h]{0.43\linewidth}

        \textbf{Summary of Variants:} Figure 4 summarizes the \textbf{different
        variants} used in the experiment. Each variant is defined by whether the
        digit recognizer and program head are pretrained and/or frozen. ``P-D''
        and ``F-D'' refer to pretraining and freezing the digit recognizer;
        ``P-H'' and ``F-H'' refer to pretraining and freezing the program head.

    \end{minipage}%
    \begin{minipage}[h]{0.51\linewidth}
        \centering
        \begin{tabular}{lcccc}
        \toprule
        \textbf{Variant} & \textbf{D-Pre} & \textbf{D-Frz} & \textbf{H-Pre} & \textbf{H-Frz} \\
        \midrule
        Baseline    & \xmark & \xmark & \xmark & \xmark \\
        P-D         & \cmark & \xmark & \xmark & \xmark \\
        F-D         & \cmark & \cmark & \xmark & \xmark \\
        P-H         & \xmark & \xmark & \cmark & \xmark \\
        F-H         & \xmark & \xmark & \cmark & \cmark \\
        P-D + P-H   & \cmark & \xmark & \cmark & \xmark \\
        P-D + F-H   & \cmark & \xmark & \cmark & \cmark \\
        \bottomrule
        \end{tabular}
        Figure 4: Summary of training variants and initialization settings.
    \end{minipage}%
    }

\column{0.5}

\block{Results}{
    \includegraphics[scale=0.8,width=\linewidth]{acc_vs_variants.png}
    \includegraphics[scale=0.8,width=\linewidth]{training_time.png}
    % \begin{itemize}
    %     \item The baseline model (without any pretraining) attains a strong 97.8\% accuracy at a learning rate of 0.005, outperforming several pretrained variants.
    %     \item Using a moderately higher learning rate enables the pretrained readhead variant to surpass 80\% accuracy, ultimately achieving 91\% final accuracy.
    % \end{itemize}
}

% Conclusion
\block{Conclusion and Next Steps}{
    \begin{itemize}
        \item Pretraining \textbf{does not} improve the \textbf{maximum achievable final accuracy} compared to a carefully trained baseline.
        \item Pretraining \textbf{accelerates convergence} and \textbf{improves training stability}.
        \item \textbf{Freezing a large fraction} of model parameters even after pretraining can severely \textbf{restrict task-specific adaptation} and lead to catastrophic training failure.
        \item \textit{Next}: Explore the benefit of pre-training with other problems where end-to-end training is more challenging (\emph{e.g.}\ Arithmetics with hand-written operators).
        \item \textit{Next}: Use \textbf{weight-variance-based metrics} to monitor the internal structure of neural networks, with the goal of improving both interpretability and debugging capabilities.
    \end{itemize}
}

% Related Work
\block{Related Work}{
    This work builds on \textit{Neural TerpreT}
    \cite{gaunt_differentiable_2017,gaunt_terpret_2016}, where perception
    modules are composed with differentiable interpreters. We contrast with
    monolithic LSTM baselines by explicitly separating visual and symbolic
    reasoning.

    Symbol correctness of a sub-network is crucial for transfer learning and interpretability
    \cite{bembenek_symbol_2024}; this project similarly evaluates intermediate
    predictions alongside task performance.

    The Neural Programmer-Interpreter (NPI) \cite{reed_neural_2016} learns
    programs compositionally by calling subprograms from memory, enabling strong
    generalization from few examples. Our project similarly leverages modular
    structures to support transfer and efficient learning.

    \printbibliography[heading=none]
}


\end{columns}

\end{document}
